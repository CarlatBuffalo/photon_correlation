\section{Intensity}
\subsection{Purpose}
For T2 and T3 data, it is often useful to group records into fixed time bins and count the number of records in each bin. This program does just that, reporting the intensity at each channel for all data in a time-ordered stream. 

\subsection{Command-line syntax}
\section{Intensity}
\subsection{Purpose}
For T2 and T3 data, it is often useful to group records into fixed time bins and count the number of records in each bin. This program does just that, reporting the intensity at each channel for all data in a time-ordered stream. 

\subsection{Command-line syntax}
\begin{verbatim}Usage: intensity [options]

Version 1.6

                -h, --help: Prints this usage message.
             -V, --verbose: Print debug-level information.
             -v, --version: Print version information.
             -i, --file-in: Input filename. By default, this is stdin.
            -o, --file-out: Output filename. By default, this is stdout.
          -f, --start-time: The lower limit of time for the run; do not
                            process photons which arrive before this time.
           -F, --stop-time: The upper limit of time for the run; do not
                            process photons which arrive before this time.
                -m, --mode: TTTR mode (t2 or t3) represented by the data.
            -c, --channels: The number of channels in the signal.
           -w, --bin-width: The width of the time bin for processing 
                            photons. For t2 mode, this is a number of
                            picoseconds, and for t3 mode this is a 
                            number of pulses.
           -A, --count-all: Rather than counting photons in a given time bin,
                            count all photons in the stream.

Given a stream of TTTR data formatted like the output of picoquant, this
program calculates the number of photons arriving on any number of detection
channels, divided into some number of subsets of integration time.

For example, to calculate the intensity of a stream of t2 photons from a
measurement on a Picoharp, collected in 50ms bins:
    intensity --bin-width 50000000000 --mode t2 --channels 2

The output format is:
    time start, time stop, channel 0 counts, channel 1 counts, ...

Because the final photon may not arrive at the end of a time bin, the final
bin ends at the arrival time of the last photon, permitting post-processing
to determine whether the effect of that edge is significant.

As an alternative to time bins, all of the photons can be counted by passing
the flag --count-all. This is useful for normalizing a signal.
\end{verbatim}

\subsubsection{Input}
T2 and T3 modes accept data of the form produced by \program{picoquant}, as specified in section~\ref{sec:picoquant_output}.

\subsubsection{Output}
The output for both T2 and T3 modes is of the form:
\begin{verbatim}
lower time limit, upper time limit, 
  channel 0 intensity, channel 1 intensity, ... \n
\end{verbatim}
For T2 mode, the time bin is defined as a number of picoseconds. For T3 mode, the time bin is a number of pulses.

Passing the flag \texttt{--count-all} will ignore the time bins, and instead count all events on each channel as belonging to one large channel. 

The bin limits for the first and last bin are set to the first and last photons seen, unless the flag \texttt{--start-time} or \texttt{--stop-time} is passed. This eliminates dead time at the beginning and end of the measurement, though for some purposes the limits may need to be extended. For such limit modifications, use external analysis code.

\subsection{Examples of usage}
\subsubsection{Bin counts into fixed-width bins}
\begin{verbatim}
> picoquant --file-in data.pt2 --number 100000 |  \
  intensity --mode t2 --channels 2 \
            --bin-width 50000000000
7128264,50000000000,2674,0
50000000000,100000000000,2715,0
100000000000,150000000000,2650,0
150000000000,186537236212,1961,0
\end{verbatim}

\subsubsection{Count all photons}
\begin{verbatim}
> picoquant --file-in data.pt2 --number 100000 |  \
  intensity --mode t2 --channels 2 \
  --count-all
0,186537236212,10000,0
\end{verbatim}

\subsubsection{Count photons in a subset of time}
\begin{verbatim}
> picoquant --file-in data.pt2 --number 100000 | \
  intensity --mode t2 --channels 2 \
  --bin-width 50000000000 
  --start-time 100000000000
100000000000,150000000000,2650,0
150000000000,186537236212,1961,0
> picoquant --file-in data.pt2 --number 100000 | \
  intensity --mode t2 --channels 2 \
  -bin-width 50000000000 
  --start-time 100000000000
  --stop-time 120000000000
100000000000,120000000000,1040,0
\end{verbatim}

\subsection{Implementation details}
\label{sec:intensity_implementation}
Each channel can be treated independently, so we will focus on how to handle a single stream of records. 

Given a set of photons $\photon\in\photons$, our goal is to determine the number of $\photon$ whose arrival times are in a time range $\timewindow$:
\begin{equation}
I(\timewindow) = \abs{\setbuilder{\photon}{\photon\in\photons;~\Time(\photon)\in\timewindow}}
\end{equation}
In our case, we will only be concerned with time intervals $\timewindow$ which are consecutive and collectively span the full integration time:
\begin{align}
\timewindow_{j} &= \left[[\timewindow_{j}\upminus,\timewindow_{j}\upplus\right) \\
\integrationtime &= \bigcup\limits_{j=0}^{m}{\timewindow_{j}}
\end{align}
As such, any photon can belong to exactly one subset $\photons_{\timewindow_{j}}$, and these $\timewindow_{j}$ can each be visited exactly once by iterating through times in the experiment. This can be performed efficiently if we impose the condition that the stream of photons be time-ordered, by iterating over photons and time bins alternately:
\lstset{language=Python}
\begin{lstlisting}
photon = next(photons)
time_bin = next(time_bins)
intensity = 0

while photon and time_bin:
    if photon in time_bin:
        intensity += 1
        photon = next(photon)
    else:
        yield(time_bin, intensity)
        time_bin = next(time_bins)
        intensity = 0

yield(time_bin, intensity)
\end{lstlisting}
As can be seen, this algorithm scales linearly with the number of time bins and photons, i.e. it scales as O(\abs{\photons}). This algorithm requires time-ordered photons and bins which never lag behind the photon stream, which are guaranteed by appropriate initialization of the photon and bin streams.



\subsubsection{Input}
T2 and T3 modes accept data of the form produced by \program{picoquant}, as specified in section~\ref{sec:picoquant_output}.

\subsubsection{Output}
The output for both T2 and T3 modes is of the form:
\begin{verbatim}
lower time limit, upper time limit, 
  channel 0 intensity, channel 1 intensity, ... \n
\end{verbatim}
For T2 mode, the time bin is defined as a number of picoseconds. For T3 mode, the time bin is a number of pulses.

Passing the flag \texttt{--count-all} will ignore the time bins, and instead count all events on each channel as belonging to one large channel. 

The bin limits for the first and last bin are set to the first and last photons seen, unless the flag \texttt{--start-time} or \texttt{--stop-time} is passed. This eliminates dead time at the beginning and end of the measurement, though for some purposes the limits may need to be extended. For such limit modifications, use external analysis code.

\subsection{Examples of usage}
\subsubsection{Bin counts into fixed-width bins}
\begin{verbatim}
> picoquant --file-in data.pt2 --number 100000 |  \
  intensity --mode t2 --channels 2 \
            --bin-width 50000000000
7128264,50000000000,2674,0
50000000000,100000000000,2715,0
100000000000,150000000000,2650,0
150000000000,186537236212,1961,0
\end{verbatim}

\subsubsection{Count all photons}
\begin{verbatim}
> picoquant --file-in data.pt2 --number 100000 |  \
  intensity --mode t2 --channels 2 \
  --count-all
0,186537236212,10000,0
\end{verbatim}

\subsubsection{Count photons in a subset of time}
\begin{verbatim}
> picoquant --file-in data.pt2 --number 100000 | \
  intensity --mode t2 --channels 2 \
  --bin-width 50000000000 
  --start-time 100000000000
100000000000,150000000000,2650,0
150000000000,186537236212,1961,0
> picoquant --file-in data.pt2 --number 100000 | \
  intensity --mode t2 --channels 2 \
  -bin-width 50000000000 
  --start-time 100000000000
  --stop-time 120000000000
100000000000,120000000000,1040,0
\end{verbatim}

\subsection{Implementation details}
\label{sec:intensity_implementation}
Each channel can be treated independently, so we will focus on how to handle a single stream of records. 

Given a set of photons $\photon\in\photons$, our goal is to determine the number of $\photon$ whose arrival times are in a time range $\timewindow$:
\begin{equation}
I(\timewindow) = \abs{\setbuilder{\photon}{\photon\in\photons;~\Time(\photon)\in\timewindow}}
\end{equation}
In our case, we will only be concerned with time intervals $\timewindow$ which are consecutive and collectively span the full integration time:
\begin{align}
\timewindow_{j} &= \left[[\timewindow_{j}\upminus,\timewindow_{j}\upplus\right) \\
\integrationtime &= \bigcup\limits_{j=0}^{m}{\timewindow_{j}}
\end{align}
As such, any photon can belong to exactly one subset $\photons_{\timewindow_{j}}$, and these $\timewindow_{j}$ can each be visited exactly once by iterating through times in the experiment. This can be performed efficiently if we impose the condition that the stream of photons be time-ordered, by iterating over photons and time bins alternately:
\lstset{language=Python}
\begin{lstlisting}
photon = next(photons)
time_bin = next(time_bins)
intensity = 0

while photon and time_bin:
    if photon in time_bin:
        intensity += 1
        photon = next(photon)
    else:
        yield(time_bin, intensity)
        time_bin = next(time_bins)
        intensity = 0

yield(time_bin, intensity)
\end{lstlisting}
As can be seen, this algorithm scales linearly with the number of time bins and photons, i.e. it scales as O(\abs{\photons}). This algorithm requires time-ordered photons and bins which never lag behind the photon stream, which are guaranteed by appropriate initialization of the photon and bin streams.

