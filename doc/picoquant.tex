\documentclass{article}

\usepackage[squaren]{SIunits}
\usepackage{amsmath,amsfonts}
\usepackage{appendix}
\usepackage{enumerate}

\newcommand{\cps}{\textnormal{cps}}
\newcommand{\braces}[1]{\ensuremath{\left\lbrace #1 \right\rbrace}}
\newcommand{\angles}[1]{\ensuremath{\left\langle #1 \right\rangle}}
\newcommand{\stdin}{\texttt{stdin}}
\newcommand{\stdout}{\texttt{stdout}}
\newcommand{\stderr}{\texttt{stderr}}
\newcommand{\picoquant}{\texttt{picoquant}}
\newcommand{\intensity}{\texttt{intensity}}
\newcommand{\correlate}{\texttt{correlate}}
\newcommand{\histogram}{\texttt{histogram}}
\newcommand{\gn}[1]{\ensuremath{g^{(#1)}}}
\newcommand{\integers}{\ensuremath{\mathbb{Z}}}
\newcommand{\wholes}{\ensuremath{\mathbb{N}}}
\newcommand{\reals}{\ensuremath{\mathbb{R}}}
\renewcommand{\vec}{\mathbf}
\newcommand{\abs}[1]{\ensuremath{\left|#1\right|}}
\newcommand{\eV}{\textnormal{eV}}
\newcommand{\channel}{\ensuremath{c}}
\newcommand{\channels}{\ensuremath{C}}
\newcommand{\Channel}{\ensuremath{\mathcal{C}}}
\newcommand{\Time}{\ensuremath{\mathcal{T}}}
\newcommand{\photon}{\ensuremath{\gamma}}
\newcommand{\photons}{\ensuremath{\Gamma}}
\newcommand{\Pulse}{\ensuremath{\mathcal{P}}}
\newcommand{\integrationtime}{\ensuremath{\Xi}}
\newcommand{\timewindow}{\ensuremath{\xi}}
\newcommand{\resolution}{\ensuremath{\epsilon}}
\newcommand{\Index}{\ensuremath{\mathcal{J}}}
\newcommand{\Histogram}{\ensuremath{\mathcal{H}}}
\newcommand{\ceil}[1]{\ensuremath{\left\lceil #1\right\rceil}}
\newcommand{\GN}{\texttt{gn}}

\title{Correlate: Tools for performing $n$th order correlations of photon arrivals}
\author{Thomas Bischof \\ \texttt{tbischof@mit.edu}}
\date{\today}

\begin{document}

\maketitle
\tableofcontents
 
\section{Introduction}
\subsection{The layout of this document}
This document is laid out in roughly three parts:
\begin{enumerate}
\item Overview of terminology and methods
\item Documentation for each program
\item Applications of the software to real problems
\end{enumerate}
In the chapters devoted to the various programs, the documentation is divided further:
\begin{enumerate}
\item Command-line syntax
\item Theoretical overview of the purpose of the program
\item Details of the implementation
\end{enumerate}

\subsection{A (very) brief overview of photon-arrival timing}
In single-molecule spectroscopy, single-photon detectors are often used to perform time-resolved experiments. Detectors such as silicon-based avalanche photodiodes (APDs) can be used to detect the arrival time a photon with about 500\pico\second{} resolution, and can detect up to $10^{7}$ photons per second before saturation, abbreviated as 10\mega\cps{} hereafter (counts per second). Hardware capable of resolving these arrival times is of great use for revealing time-dependent structure to the photon stream, such as intensity fluctuations and bunching, so various hardware designs have been developed to permit such measurements.

One line of instruments is the timing hardware produced by Picoquant GmBH, such as the Timeharp, Picoharp, and Hydraharp. These modules are capable of detecting pulse arrivals on multiple input channels with a resolution of as little as 1\pico\second, and operate in a few distinct modes:
\begin{enumerate}
\item Interactive (histogram): one input channel is designated as a sync source, representing a clock-starting signal. On the other channels, pulse arrival times are recorded relative to this clock source, and the times binned into a histogram. 
\item Time-tagged time-resolved (TTTR): 
	\begin{enumerate}
	\item T2: all channels are treated equally, and all pulse arrival times relative to the start of the experiment are recorded. 
	\item T3: this is similar to histogram mode, but instead of binning the arrival times, the sync event number and relative arrival time are both recorded.
	\end{enumerate}
\end{enumerate}

In all cases, the times are discrete and represent some number of cycles of a clock, so all times must be treated as integers representing some number of picoseconds. This has important effects on the definition of time bins for histograms, and where appropriate some time will be devoted to discussing these factors. 

Considering these distinct modes, it is important to spend some time discussing their uses, and how their data should be handled.

\subsection{Data collection modes}
\label{sec:modes}
\subsubsection{Interactive (histogram)}
A common experiment for studying fluorophores is to measure the time dependence of their response to an excitation. For example, a pulsed laser can be used to excite a sample, and the resulting fluorescence detected by an APD. Because single-photon detectors are limited to detection of a single photon at once, to reconstruct the decay curve it is necessary to average the result over many pulses of the laser, binning the arrival times of the emitted photons relative to that of the laser. If measuring this time-averaged behavior is sufficient, then the interactive histogram mode is used to perform all of this collection and binning on the hardware without any post-processing. 

Consequently, interactive data consists of $N$ unique arrival time bins with boundaries $(b_{j}, b_{j+1})$ and the number of counts $n_{j}$ associated with that bin, so the data can be represented by the set:
\begin{equation}
\braces{((b_{j}, b_{j+1}), n_{j})}
\end{equation}
The exact choice of where the boundaries lie has some effect on the resulting histogram, but this performed in the hardware and presumably represents $t\in[b_{j}, b_{j+1})$. This choice is not detailed in the manuals for the hardware, but it is of little practical importance; typical detectors have two orders of magnitude more timing jitter than the timing hardware, so the exact definition of the histogram bin has a negligible effect.

\subsubsection{T2}
In T2 mode, all input channels are connected to photon detectors. At the start of the experiment, an internal clock is reset and started, providing a master timing reference. As a pulse arrives, the machine emits data encoding the channel and time of arrival, so the data follow the form:
\begin{equation}
\braces{(c, t)}
\end{equation}
for a channel $c$ and arrival time $t$ of a photon.
%Typically, these data are used to examine time-dependent behavior, such as intensity fluctuations, particularly when the excitation source is continuous-wave. These data are also useful for calculation of correlation functions such as:
%\begin{equation}
%\gn{n}(\tau_{1}, \ldots \tau_{n-1}) = \frac
%	{\angles{I_{0}(t)\prod_{j=1}^{n-1}{I_{j}(t+\tau_{j})}}}
%	{\angles{I_{0}(t)}\prod_{j=1}^{n-1}{\angles{I_{j}(t+\tau_{j})}}}
%\end{equation}
%The details of this calculation will laid out later in section~\ref{sec:correlate}, but one use of the $g^{(2)}(t)$ is to determine the number of emitters present in a signal. For example, a single emitter emitting one photon at any time will exhibit so-called antibunching behavior, where $g^{(2)}(t)\rightarrow 0$ for $t\approx 0$, indicating a diminished probability of seeing two successive photon emissions ($g^{(n)}(\braces{t_{j}})=1$ indicates no correlation). Correlations of higher order ($n\ge 3$) have their own uses, but quickly become computationally expensive for reasons which will become clear later.

\subsubsection{T3}
T3 mode is closest in character to the interactive mode, except that, instead of binning the photon arrival times on the hardware, the arrivals are recorded directly for later examination, as in T2 mode. This gives a data set of the form:
\begin{equation}
\braces{(c, p, t)}
\end{equation}
for channel $c$, sync pulse number $p$, and relative arrival time $t$. Indeed, by defining temporal bins and histogramming the $t$, the result from the interactive mode can be reproduced exactly.

\paragraph{Mapping T3 onto T2 data}
\label{sec:t3_to_t2}
For signals produced by regular excitation it is possible to convert T3 data to T2 data. For example, consider a sample excited by a laser with repetition rate $f$. The pulse number therefore defines a time $1/f$, and $t$ acts as a correction to this time:
\begin{equation}
\left(c, p, t\right) \rightarrow \left(c, \frac{p}{f} + t\right)
\end{equation}
Of course, this conversion is not necessarily perfect. For example, sync pulses may be missed or overcounted, introducing a cumulative error $\delta p$, the frequency of the sync pulse might deviate with some jitter $\delta f$, or the timing of the arrival might have some jitter $\delta t$, such that the true expression is:
\begin{equation}
\left(c, p+\delta p, t + \delta t \right)
      \rightarrow \left(c, \frac{p+\delta p}{f+\delta f} + t+\delta t\right)
\end{equation}
For a well-designed system these errors should not be very large, but if an application requires precision approaching that of the timing hardware they can be quite important. 

% However, this mode also allows for studying time dependence and correlation in the data, as with T2 mode. In principle, any data collected in T3 mode can be transformed into T2 mode data if the sync source is regular, as the pulse number will represent some amount of time in the experiment, but there are subtle hardware and numerical issues (discussed later) which limit the practicality of this transformation.
%
%Uses of this mode include the study of the time-dependent fluoresence lifetime of single molecules, which can switch between distinct states under various conditions. Correlation methods can also reveal important behavior, but again this will be discussed later.

\subsection{Correlation techniques}
A number of techniques involving the correlation of one or more signals exist, and their use in fluorescence microscopy requires that we be able to calculate correlation functions of the following form:
\begin{equation}
\gn{n}(\tau_{1}, \ldots \tau_{n-1}) = \frac
	{\angles{I_{0}(t)\prod_{j=1}^{n-1}{I_{j}(t+\tau_{j})}}}
	{\angles{I_{0}(t)}\prod_{j=1}^{n-1}{\angles{I_{j}(t+\tau_{j})}}}
\end{equation}
The details of calculating this function will be laid out in section~\ref{sec:math_background}. For now, note that the problem can be divided into three distinct parts:
\begin{enumerate}
\item Calculation of \angles{I_{j}(t)}
\item Finding correlation events
\item Histogramming correlation events
\end{enumerate}

\subsection{General design principles}
Analysis of these timing data can be roughly described as follows:
\begin{enumerate}
\item Produce a stream of photon events
\item Condition the stream by correlation, removal of extraneous information, homogenization of the detection channels, etc.
\item Collect the result to form some histogram of events.
\end{enumerate}
As such, these three phases are handled by distinct programs, which act as filters of a data stream by reading in values and outputting the appropriate new values. These streams are ultimately streams of binary data, but for any program the stream may be defined by the standard data streams (\stdin, \stdout) or by some file containing the data. 

For time-tagged modes, times are represented as integer multiples of 1\pico\second, although only the initial data streamer is actually aware of units; the conditioning and collection routines operate on time as an integer, without regard for its units.

All programs are designed to operate on a data stream until that stream terminates, so any division of a data stream should be handled by a separate program and the result fed into a distinct instance of the handler. 

The main processing software (\picoquant, \intensity, \correlate, \histogram) is written in the C programming language, using the C99 standard. Where necessary, the definitions of data types have been defined to be of fixed width, but otherwise the definitions are those minimal for handling reasonable values. For example, in T2 mode time is defined as a signed 64-bit integer, representing $\approx 10^{19}\pico\second$, or $10^{7}\second$, or $106$ days. Pulses are counted as signed 64-bit integers, limiting their total to several thousand years of 10\mega\hertz{} pulsed laser excitation. With luck, your experiments will not exceed these values. 

In addition to the C code, several analysis scripts written in Python have been provided in \text{scripts/}. These include routines to read in the outputs of most of the processing software, generate correlation functions, plot lifetime data, and so on. 

All programs will display command-line syntax when the flag \texttt{-h} or \texttt{--help} is given. The C programs have been tested most thoroughly on 32-bit and 64-bit Linux systems with the GNU C compiler, but in principle should work on any system with a C99-compliant compiler. Non-standard libraries are not used, to limit portability issues. A Makefile is present in the \texttt{src} directory which should be sufficient for compiling the software, although it will require some modification for different compiler configurations.

The Python programs have been tested primarily with version 2.7 on the same systems described above, but should also work in versions 3.x.

\section{The mathematics of calculating \gn{n}}
\label{sec:math_background}
This section will detail the mathematics of calculating correlation functions \gn{n}. If you are familiar with the results, you can safely skip this section, but do note that the correlation of photon events is in many ways distinct from a standard signal correlation.

If you are not familiar with notation such as
\begin{align}
c &= \braces{(a,b)|a,b\in\integers;~a/b\in\integers} \\
F&:\wholes\times\integers\rightarrow\reals \\
&\sum_{z\in\integers_{n}}{z^{2}} \\
&A\cap B
\end{align}
you should read appendix~\ref{sec:notation} before reading the remainder of this section.

\subsection{Definition of the correlation function}
\label{sec:correlation_function}
A signal $I(t)$ is a non-negative, real-valued function of whole time\footnote{In principle, we could all the function to return negative numbers without mathematical problems, but this affects our interpretation of the result and is more natural for the problems laid out in this paper.}:
\begin{equation}
I:\wholes\rightarrow\reals^{*}
\end{equation}
In practice such a function may represent a physical quantity such as voltage, and we will develop our understanding of photon correlations by developing first an understanding of how to correlate this sort of function. 

The simplest non-trivial correlation of a function is the autocorrelation, which measures the average probability that the value of the function for a time $t+\tau$ will be greater than, less than, or equal to its value at $t$. This can be calculated as:
\begin{equation}
\label{eq:autocorrelation}
\gn{2}(\tau) = \frac{\angles{I(t)I(t+\tau)}}
                    {\angles{I(t)}\angles{I(t+\tau)}}
\end{equation}
Here, $\gn{2}>1$ indicates supercorrelation, that the function is more likely than not to increase in value after a time delay $\tau$. If $\gn{2}<1$, the function is more likely than not to decrease in value after a time delay $\tau$. And in the middle, if $\gn{2}=1$, the function has equal probability to increase, decrease, or remain constant. The meaning of the exact magnitude of $\gn{2}(\tau)$ can be discussed in the context of a particular function, but these general principles should always apply.

In many cases, an autocorrelation is an oversimplification of a signal, and it is useful to be able to compare the cross-correlations of two or more channels instead. For example, if our function $I$ is endowed with a second dimension indicating the identity of a detection channel from the set $C$ of detection channels, it takes the form
\begin{equation}
I:C\times\wholes\rightarrow\reals^{*}
\end{equation}
where $\times$ is the Cartesian product of elements from $C$ and $\integers^{*}$. For example, if $C=\braces{0, 1}$, elements of $C\times\wholes$ take the form of 2-tuples such as (0, 10), (1, 17), (0, 0), and so on. Under this notation, $I$ is now a function of two variables and follows the form $I(c, t)$. For clarity of future notation, we will include the channel variable as a subscript:
\begin{equation}
I(c,t)\equiv I_{c}(t)
\end{equation}
Given this, we can separate the signal $I$ into a sum of signals over all channels:
\begin{equation}
I(t) \equiv \sum_{c\in C}{I_{c}(t)}
\end{equation}
Under the definition of an autocorrelation as in equation~\ref{eq:autocorrelation}, we can substitute this new value and expand the result to obtain:
\begin{equation}
\gn{2}(\tau)=\sum_{(c_{0},c_{1})\in C^{2}}{\frac{\angles{I_{c_{0}}(t)I_{c_{1}}(t+\tau)}}
                                                {\angles{I_{c_{0}(t)}}\angles{I_{c_{1}}(t+\tau)}}}
\end{equation}
where $(c_{0},c_{1})\in C^{2}$ indicates 2-tuple elements of the set $C\times C$. We see that, if $C$ contains a single element (we have only one detection channel), the function returns to the form shown in equation~\ref{eq:autocorrelation}. But if we have a set $C$ with more than one element, there are more than one term in the summation, which each take the form
\begin{equation}
\gn{2}_{(c_{0},c_{1})}(\tau)=\frac{\angles{I_{c_{0}}(t)I_{c_{1}}(t+\tau)}}
                                                {\angles{I_{c_{0}(t)}}\angles{I_{c_{1}}(t+\tau)}}
\end{equation}
We will focus most of our attention on the individual cross-correlations, because these contain more specific information than the full autocorrelation found by summation over all channel combinations ($\vec{c}\equiv (c_{0}, c_{1},\ldots)\in C^{n}$). 
As an aside, note that the interchange of two channels is equivalent to inversion of their relative time delay:
\begin{align}
\gn{2}_{(c_{0},c_{1})}(\tau)&=\frac{\angles{I_{c_{0}}(t)I_{c_{1}}(t+\tau)}}
                                                {\angles{I_{c_{0}}(t)}\angles{I_{c_{1}}(t+\tau)}} \\
                            &=\frac{\angles{I_{c_{0}}(t-\tau)I_{c_{1}}(t)}}
                                                {\angles{I_{c_{0}}(t-\tau)}\angles{I_{c_{1}}(t)}} \\
                            &= \gn{2}_{(c_{1}, c_{0})}(-\tau)
\end{align}
This relationship implies that, while cross-correlations may be asymmetric about $\tau=0$, the full autocorrelation must be symmetric. 

Now that we have generalized this definition to an arbitrary number of channels, we can also generalize the correlation to an arbitrary order $n$:
\begin{equation}
\label{eq:gn}
\gn{n}(\tau_{1},\ldots\tau_{n-1}) = \sum_{\vec{c}\in C^{n}}{
     \frac{\angles{I_{c_{0}}(t)\prod_{j=1}^{n-1}{I_{c_{j}}(t+\tau_{j})}}}
          {\angles{I_{c_{0}}(t)}\prod_{j=1}^{n-1}{\angles{I_{c_{j}}(t+\tau_{j})}}}} 
\end{equation}

\subsubsection{Examples of correlations of functions}
To become more familiar with the behavior of these correlation functions, we should evaluate them for a few familiar functions. For example, for $I(t)=1+\sin{(t)}$:
\begin{align}
\gn{2}(\tau) &= \frac{\frac{1}{2\pi}\int_{-\pi}^{\pi}{\left(1+\sin{(t)}\right)\left(1+\sin{(t+\tau)}\right)\,dt}}
                     {\left(\frac{1}{2\pi}\int_{-\pi}^{\pi}{\left(1+\sin{(t)}\right)\,dt}\right)
                      \left(\frac{1}{2\pi}\int_{-\pi}^{\pi}{\left(1+\sin{(t+\tau)}\right)\,dt}\right)} \\
             &= 1 + \frac{1}{2}\cos{(\tau)}
\end{align}
For, this result says that, given that our function at some value at time $t$, after a small time delay $\tau\approx 0$ the value is likely to have increased ($\gn{2}=1.5$). If you were expecting $\gn{2}(0)$ to be 1, note that the correlation function is not strictly an increase or decrease of the value of the function, but instead more akin to a weighted average of that character. Thus, while the function has intervals of equal size over which it increases ($t\pmod{2\pi} \in (0,\frac{\pi}{2})\cup(\frac{3\pi}{2})$) or decreases ($t\pmod{2\pi} \in (\frac{\pi}{2},\frac{3\pi}{2})$), the time it spends with value greater than average ($t\pmod{2\pi}\in(0,\pi)$) weighs more heavily. 

Additionally, at $\tau=2\pi$, the value of the function is exactly what it was at $\tau=0$, so $\gn{2}(2\pi)=1$.

Moving on, consider a signal composed of two Gaussian functions:
\begin{align}
I_{0}(t) &= \frac{1}{\sigma_{0}\sqrt{2\pi}}\exp{\left(-(t-\mu_{0})^{2}/(2\sigma_{0}^{2})\right)} \\
I_{1}(t) &= \frac{1}{\sigma_{1}\sqrt{2\pi}}\exp{\left(-(t-\mu_{1})^{2}/(2\sigma_{1}^{2})\right)} \\
I(t) &= I_{0}(t)+I_{1}(t)
\end{align}
We can calculate the full autocorrelation as the sum of the cross-correlations:
\begin{align}
\gn{2}(\tau) &= \gn{2}_{(0,0)}(\tau)+\gn{2}_{(0,1)}(\tau)+\gn{2}_{(1,0)}(\tau)+\gn{2}_{(1,1)}(\tau) \\
             &= \frac{\exp{\left(-\frac{\tau^{2}}{4\sigma_{0}^2}\right)}}{2\sigma_{0}\sqrt{\pi}}
                + \frac{\exp{\left(-\frac{(\mu_{0}-\mu_{1}+\tau)^{2}}{2(\sigma_{0}^{2}+\sigma_{1}^{2})}\right)}}
                        {\sqrt{2\pi(\sigma_{0}^{2}+\sigma_{1}^{2})}} \nonumber \\
                &~~~~+ \frac{\exp{\left(-\frac{(\mu_{0}-\mu_{1}-\tau)^{2}}{2(\sigma_{0}^{2}+\sigma_{1}^{2})}\right)}}
                        {\sqrt{2\pi(\sigma_{0}^{2}+\sigma_{1}^{2})}}
                +
                \frac{\exp{\left(-\frac{\tau^{2}}{4\sigma_{1}^2}\right)}}{2\sigma_{1}\sqrt{\pi}}
\end{align}
This is a somewhat complicated expression, but we see that there are two Gaussians centered at $\tau=0$, as well as two Gaussians centered at $\abs{\mu_{0}-\mu_{1}}$ on either side of the origin. 

Calculation of higher-order correlations is also possible, but the results quickly become verbose and are not worthwhile presenting here.

\subsection{The true meaning of $I(t)$}
\label{sec:sampling_intensity}
For a real measurement, the signal $I(t)$ is real-valued and defined by averaging some signal for a time interval $\Delta t$, such that the value $I(t)$ really is
\begin{equation}
I(t) = \left.\angles{\iota(t')}\right|_{t'\in[t,t+\Delta t)} = \frac{1}{\Delta t}\int_{t}^{t+\Delta t}{\iota(t'),dt'}
\end{equation}
for the true function $\iota(t)$ being approximated by the measurement. As such, the $I(t)$ over time can be represented meaningfully as a vector representing the value of the function for evenly-spaced values of $t$, and the correlations can be determined as inner products of displacements of that vector. More concretely, consider a signal $\vec{I}\in\left(\reals^{*}\right)^{N}$ representing $N$ samples of $I(t)$ at $t=0, \Delta t, \ldots$, with elements indexed as $\vec{I}(0), \vec{I}(1),\ldots$. To calculate $\gn{2}(\tau)$ for $\tau\in\wholes$:
\begin{equation}
\gn{2}(\tau) = \frac{\sum_{j=0}^{N-\tau}{\vec{I}(j)\vec{I}(j+\tau)}}
                    {\sum_{j=0}^{N-\tau}{\vec{I}(j)}\sum_{j=0}^{N-\tau}{\vec{I}(j+\tau)}}
\end{equation}
Note that, for $\tau\rightarrow N$, the number of elements in the sum approaches 0. This represents the undersampled region of the correlation, and as such it is necessary to define the correlation window as significantly smaller than the sampled window in order to obtain a meaningful estimate of $\gn{2}(\tau)$.

This definition of a signal is useful for many measurements of some gross quantity which can be said to sample a non-trivial range of values in $\reals^{*}$ or \wholes. However, photon-counting produces a signal which is fundamentally binary (in $\integers_{2}$), indicating that either a photon has arrived in the time interval, or none has. In principle we can treat this vector in the same way as we do $\vec{I}$, but this is inefficient: there are only a few bins which will have any signal, and a great number which contain nothing. Therefore, for a binary signal which only occasionally has a non-zero value, it is worthwhile to develop different forms for the correlation expressed in equation~\ref{eq:gn}.

\subsection{Defining the signal of photon arrivals}
In practice, any given single-photon detector can detect exactly one photon at a time, such that any photon arrival can be defined unique by its detection channel and arrival time. Therefore, the photon $\gamma$ can be represented as a 2-tuple:
\begin{equation}
\photon\equiv (c, t)\in C\times\wholes
\end{equation}
We define the functions $\Channel(\photon)$ and $\Time(\photon)$ to return the channel and arrival time of a photon \photon. From this definition, call
\begin{equation}
\photons\equiv\braces{\photon}\subset C\times\wholes
\end{equation}
the set of all detected photons. This is the signal relevant to photon-correlation methods, and its form requires that we recast the correlation function in a way which is more directly applicable. 

To begin with, the definition of the signal on a single channel can be expressed as:
\begin{equation}
\photons_{c} = \braces{\photon\left|\photon\in\photons; \Channel(\photon)=c\right.}
\end{equation}
In long form, this signal is the set of all detected photons, restricted to those whose channels matches the channel specified (see appendix~\ref{sec:notation} for more details). Here, there is no longer an explicit time dependence, but we can define that with an additional parameter:
\begin{equation}
\photons_{c}(t) = \abs{\braces{\photon\left|\begin{aligned}
                                          \photon\in\photons; \\
                                           \Channel(\photon)=c;\\
                                           \Time(\photon)=t
                                     \end{aligned}\right.}}
\end{equation}
where here the \abs{\cdot} indicate the number of elements in the set. To recover the complete signal \photons:
\begin{equation}
\photons = \bigcup\limits_{c\in C}{\photons_{c}}
\end{equation}

\subsubsection{T3 mode is akin to T2 mode, but a second time dimension}
As discussed in section~\ref{sec:modes}, T3 mode has a definition distinct from that of T2 mode:
\begin{equation}
\photon\equiv(c,p,t)\in C\times\wholes\times\wholes
\end{equation}
for which $\Pulse(\photon)$ returns the pulse number for which the photon arrived. While the definitions are not perfectly clean, we will show shortly how T3 photons can be treated exactly like a pair of T2 photons, with some restrictions.

\subsection{Calculating \gn{n} by counting photons}
As before, we will begin our discussion of correlation functions by constructing a set formulation of the autocorrelation. Fundamentally, we must consider two halves of a problem:
\begin{equation}
\gn{n}((\tau_{1},\ldots))=\frac{\textnormal{number of events which satisfy a given time delay}}
     {\textnormal{average number of photons per unit of time}}
\end{equation}
The denominator is simpler to express, so we will start there. In a given experiment of finite length, there will be some absolute beginning and end of time, such that all detected photons are elements of the subset
\begin{equation}
C\times N_{\integrationtime}
\end{equation}
for a total experiment time \integrationtime. That is, there exists some time $\integrationtime\in\wholes$ such that
\begin{equation}
\braces{\photon\left|\photon\in\photons;\Time(\photon)\ge\integrationtime\right.} = \emptyset
\end{equation}
Here, the integration time is defined such that there are \integrationtime{} time units which pass during the experiment, since time begins with 0. This sets the minimal value for \integrationtime{} to be one greater than the arrival time of the final photon in the experiment, or
\begin{equation}
\integrationtime\ge\max\braces{\Time(\photon)|\photon\in\photons}
\end{equation}
Now that we have determined the total units of time represented by \photons{} (either by a defined value or by some maximal time \Time(\photon)+1), the average number of photons arriving per unit of time is
\begin{equation}
\angles{I(t)} = \frac{\abs{\photons}}{\integrationtime}
\end{equation}
Next, consider the set of photons satisfying some specified time delay $\tau$:
\begin{equation}
\angles{I(t)I(t+\tau)}\propto
         \abs{\braces{(\photon_{0},\photon_{1})
               \left|\begin{aligned}
                     \photon_{0},\photon_{1}\in\photons;\\
                     \Time(\photon_{1})-\Time(\photon_{0})=\tau
                     \end{aligned}\right.}}
\end{equation}
This is accurate up to a normalization constant, which is related to the resolution of \Time. As discussed in section~\ref{sec:sampling_intensity}, the function does not have infinite resolution but instead approximates some true function by sampling for an interval we will call \resolution. As such, the set defined is actually for a square in time space with length \resolution{} ($t$ and $\tau$ are really $[t,t+\resolution)$ and $[\tau,\tau+\resolution)$), so we must correct for this value to be completely general:
\begin{equation}
\angles{I(t)I(t+\tau)}=    
         \frac{1}{\resolution^{2}}
         \abs{\braces{(\photon_{0},\photon_{1})
               \left|\begin{aligned}
                     \photon_{0},\photon_{1}\in\photons;\\
                     \Time(\photon_{1})-\Time(\photon_{0})=\tau
                     \end{aligned}\right.}}
\end{equation}
For the most precise form of this calculation, $\resolution=1$, but for practical reasons it will become necessary to undersample the correlation function by increasing the effective value of \resolution. This resolution should also be allowed to vary with $\tau$, so we will denote its full behavior $\resolution_{c}(\tau)$, where the $c$ subscript indicates its associated channel. Note that the reference channel $\channel_{0}$ does not have a varying resolution, because it has not associated time delay (see equation~\ref{eq:gn}). See section~\ref{sec:histogram} for more details.

Putting this together, the full autocorrelation is
\begin{equation}
\gn{2}(\tau) = \frac{\integrationtime^{2}}{\abs{\photons}^{2}\resolution\resolution(\tau)}
                \abs{\braces{(\photon_{0}, \photon_{1}),
                              \left|\begin{aligned}
                              \photon_{0},\photon_{1}\in\photons \\
                              \Time(\photon_{1})-\Time(\photon_{0})=\tau
                              \end{aligned}\right.}}
\end{equation}
Extension of this result to higher dimensions and multiple channels proceeds much as before, giving the following two-channel cross-correlation:
\begin{equation}
\gn{2}(\tau) = \sum\limits_{(c_{0},c_{1})\in C^{2}}
                    {
                    \frac{\integrationtime^{2}}
                         {\abs{\photons_{c_{0}}}\abs{\photons_{c_{1}}}
                                \resolution_{c_{0}}\resolution_{c_{1}}(\tau)}
                    \abs{\braces{(\photon_{0},\photon_{1})
                          \left|\begin{aligned}
                          \photon_{0}\in\photons_{c_{0}};\\
                          \photon_{1}\in\photons_{c_{1}};\\
                          \Time(\photon_{1})-\Time(\photon_{0})=\tau
                          \end{aligned}\right.}}
                    }
\end{equation}
and this result for higher dimensions:
\begin{equation}
\label{eq:gn_set}
\begin{split}
&\gn{n}(\tau_{1},\ldots\tau_{n-1})= \\
& \sum\limits_{\vec{c}\in C^{n}}
                    {
                    \left[
                    \left(
                    \frac{\integrationtime}{\abs{\photons_{c_{0}}}\resolution_{c_{0}}}
                    \prod_{j=1}^{n-1}{\frac{\integrationtime}
                                           {\abs{\photons_{c_{j}}}\resolution_{c_{j}}(\tau_{j})}}
                    \right)
                    \abs{\braces{(\photon_{0},\ldots\photon_{n-1})
                          \left|\begin{aligned}
                          \photon_{0}\in\photons_{c_{0}};\\
                          \photon_{1}\in\photons_{c_{1}};\\
                          \ldots;\\
                          \Time(\photon_{1})-\Time(\photon_{0})=\tau_{1};\\
                          \ldots
                          \end{aligned}\right.}}
                    \right]
                    }
\end{split}
\end{equation}

\subsubsection{The \gn{n} for T3 data can calculated like \gn{2n} for T2 data}
Given the notation in equation~\ref{eq:gn_set}, we see that the two timing dimensions of T3 data can be treated as separate conditions in the set, with a second unit of resolution associated with \Pulse. Calling this unit of resolution $\kappa$ and the relative pulse difference $\rho$, the full expression is:
\begin{equation}
\label{eq:gn_set_t3}
\begin{split}
&\gn{n}(\tau_{1},\rho_{1},\ldots\tau_{n-1},\rho_{n-1}) = \\
    &  \sum\limits_{\vec{c}\in \channels^{n}}
                    {
                    \left[
                    \frac{\integrationtime^{2}}
                         {\abs{\photons_{\channel_{0}}}\resolution_{\channel_{0}}\kappa_{\channel_{0}}}
                    \left(
                    \prod_{j=1}^{n-1}{\frac{\integrationtime^{2}}
                                           {\abs{\photons_{c_{j}}}^{2}
                                            \resolution_{c_{j}}(\tau_{j})
                                            \kappa_{c_{j}}(\tau_{j})}}
                    \right)
                    \abs{\braces{(\photon_{0},\ldots\photon_{n-1})
                          \left|\begin{split}
                          \photon_{0}\in\photons_{c_{0}};\\
                          \photon_{1}\in\photons_{c_{1}};\\
                          \ldots;\\
                          \Time(\photon_{1})-\Time(\photon_{0})=\tau_{1};\\
                          \Pulse(\photon_{1})-\Pulse(\photon_{0})=\rho_{1};\\
                          \ldots
                          \end{split}\right.}}
                    \right]
                    }
\end{split}
\end{equation}
This form is nearly identical to a \gn{2n} for T2 data, except that we still sample the channel combinations from $\channels^{n}$, reflecting the fact that $\tau_{j}$ and $\rho_{j}$ are associated with the same channel $\channel_{j}$. We will continue to discuss T2-type correlation functions, but do not that the machinery developed for such uses can easily be repurposed for T3 data.

\subsection{Subdividing the problem of calculating \gn{n}}
Examining equation~\ref{eq:gn_set}, it is evident that there are a few distinct factors associated with each term of the sum:
\begin{itemize}
\item $\lambda$: the integration time
\item $\abs{\photons_{c}}$: the number of photons with a given channel \channel.
\item $\resolution_{c}(\tau)$: the resolution of a channel at a given time delay
\item $(\photon_{0},\ldots)$: $n$-tuples of photons with particular properties
\end{itemize}
Because these factors can be calculated or defined independently, we will turn our focus to the efficient determination of the value of each of these factors. Roughly, the terms can be calculated using the following programs:
\begin{itemize}
\item $\lambda$: \intensity
\item $\abs{\photons_{c}}$: \intensity
\item $\resolution_{c}(\tau)$: \picoquant, \histogram
\item $(\photon_{0},\ldots)$: \correlate, \histogram
\end{itemize}
The remainder of this paper is devoted to specifying how each term may be calculated efficiently and accurately.

%In principle it is possible to bin these photon arrivals to count the number of arrivals in some time interval and recover the vector-like signal discussed above, but such steps introduce a range of subtle artifacts related to the precise origin of time and definition of bin resolution. These problems are largely avoidable if we instead develop a definition of \gn{n} which involves counting these events directly.

%Do note, however, that the ``true'' photon arrival time discussed from here on is itself a binary form of this vectorial definition, because real instruments will have some finite timing resolution. In this sense, the idea of a discrete arrival time is just a simplification of the true signal, where each sampling represents the state of a photon arriving or not arriving during that interval. Many more samples will find no photon than one, so those are simply not reported. Additionally, this means that the reported photon arrival time carries some time units defined by the resolution of the measurement, so we can declare any arrival time $t$ to be a multiple of these time steps, or $t\in\wholes$, where \wholes{} is the set of all whole numbers (the positive integers and zero).  
%Figure~\ref{fig:gaussian_g2} shows this result for various values of the four adjustable parameters, demonstrating the symmetric and asymmetric behavior of the various terms in the sum.
%
%\begin{figure}
%\centering
%\caption{Graphs showing different \gn{2} for a sum of two Gaussians as parameters are tuned.}
%\label{fig:gaussian_g2}
%\end{figure}

% A correlation of such a signal quantifies the randomness of its behavior over time: a signal with strong time correlation has well-defined behavior at a time $I(t+\tau)$ given a value at $I(t)$, and weaker correlation indicates that the value is less well-defined, approaching complete randomness. The correlation of a signal with itself (its autocorrelation) can be defined as:
%\begin{equation}
%\gn{2}(\tau) = \frac{\angles{I(t)I(t+\tau)}}
%                    {\angles{I(t)}\angles{I(t+\tau)}}
%\end{equation}
%where the angled brackets indicate an average over $t$. For this function, a value $\gn{2}(\tau)=1$ indicates non-correlation: at a time delay $\tau$, the value $I(t+\tau)$ is on average the same as $I(t)$. For $\gn{2}(\tau)>1$, $I(t+\tau)$ is greater than $I(t)$, and for $\gn{2}(\tau)<1$, $I(t+\tau)$ is less than $I(t)$. In terms of photon correlation methods, $\gn{2}(\tau)>1$ is called super-bunching (a photon arrival is likely to be followed by another), while $\gn{2}(\tau)<1$ is called anti-bunching (a photon arrival is likely to be followed by a lack of photons).
%
%For example, consider the autocorrelation of a sinusoid:
%\begin{align}
%\gn{2}(\tau) &= \frac{\frac{1}{2\pi}\int_{-\pi}^{\pi}{\left(1+\sin{(t)}\right)\left(1+\sin{(t+\tau)}\right)\,dt}}
%                     {\left(\frac{1}{2\pi}\int_{-\pi}^{\pi}{\left(1+\sin{(t)}\right)\,dt}\right)
%                      \left(\frac{1}{2\pi}\int_{-\pi}^{\pi}{\left(1+\sin{(t+\tau)}\right)\,dt}\right)} \\
%             &= 1 + \frac{1}{2}\cos{(\tau)}
%\end{align}
%it is evident that there is some structure to the autocorrelation, such that there is some probability of the signal being stronger or weaker at relative time delays $\tau$. 
%
%As an example which is more relevant to photon-correlation, consider a pulse train represented by
%\begin{equation}
%\label{eq:delta_train}
%I(t) = \sum_{n\in\integers}{\delta(t-n)}
%\end{equation}
%where $\delta$ here represents the mathematical delta function
%\begin{equation}
%\delta(t) = \left\lbrace \begin{split}
%                          1;~t=0 \\
%                          0;~t\not=0
%                         \end{split}
%            \right.
%\end{equation}
%This signal represents is a regularly-spaced pulse train where a single pulse arrives every unit of time, and as such its autocorrelation is:
%\begin{equation}
%\gn{2}(\tau) = \sum_{n\in\integers}{\delta(\tau-n)}
%\end{equation}
%
%
%\subsection{Extending the correlation arbitrary numbers of signals}
%While the autocorrelation of a signal is often a meaningful quantity to calculate, cross-correlations are more general and have many more applications. For example, the cross-correlation of a laser pulse train and the response of a light-emitting sample can be used to measure the lifetime of the emissive state, and is the implicit measurement of the interactive mode. 
%
%Generalization of the correlation is fairly simple: the numerator holds an average over a product of some number of signals with time delays relative to a reference channel, normalized by the average intensity at each channel. For two channels, this can be expressed as:
%\begin{equation}
%\label{eq:g2}
%\gn{2}(\tau) = \frac{\angles{I_{0}(t)I_{1}(t+\tau)}}
%                    {\angles{I_{0}(t)}\angles{I_{1}(t+\tau)}}
%\end{equation}
%Generalization to higher dimensions is relatively straightforward:
%\begin{equation}
%\label{eq:gn}
%\gn{n}(\tau_{1}, \ldots \tau_{n-1}) = \frac
%	{\angles{I_{0}(t)\prod_{j=1}^{n-1}{I_{j}(t+\tau_{j})}}}
%	{\angles{I_{0}(t)}\prod_{j=1}^{n-1}{\angles{I_{j}(t+\tau_{j})}}}
%\end{equation}
%where the $\prod$ notation indicates a product of elements, akin to the $\sum$ notation for summation.

%\subsubsection{Mapping T3 correlations onto T2-like correlations}
%One benefit of generalizing the correlation to higher dimensions is that it provides a simple way to treat correlations of T3 data as higher-dimensional T2 data. For example, if we apply the map
%\begin{equation}
%\left(c_{j}, p_{j}, t_{j}\right) \rightarrow \left(\left(c_{j}, p_{j}\right), \left(c_{j}, t_{j}\right)\right)
%\end{equation}
%it is evident that we can treat the single T3 entry as containing two dimensions of time to analyze independently. Therefore, any calculation of a correlation of T3 data can be expressed as
%\begin{equation}
%\gn{n}(\rho_{1}, \tau_{1}, \ldots \rho_{n-1}, \tau_{n-1}) = \frac
%	{\angles{I_{0}(p, t)\prod_{j=1}^{n-1}{I_{j}(p+\rho_{j},t)I_{j}(p,t+\tau_{j})}}}
%	{\angles{I_{0}(p, t)}^{2}\prod_{j=1}^{n-1}{\angles{I_{j}(p+\rho_{j},t)}
%	                                       \angles{I_{j}(p,t+\tau_{j})}}}
%\end{equation}
%For the rest of this paper, T3 data will be treated as higher-dimensional T2 data.
%
%\subsection{The true meaning of $I(t)$}
%For many measurements, the signals $I_{j}(t)$ are real-valued and defined by averaging some signal for a time interval $\Delta t$, such that the value $I_{j}(t)$ really is
%\begin{equation}
%I_{j}(t) = \left.\angles{\iota_{j}(t')}\right|_{t'\in[t,t+\Delta t)} = \frac{1}{\Delta t}\int_{t}^{t+\Delta t}{\iota_{j}(t'),dt'}
%\end{equation}
%for a the true function $\iota(t)$. As such, the $I(t)$ over time can be represented meaningfully as a vector representing the value of the function for evenly-spaced values of $t$, and the correlations can be determined as inner products of displacements of that vector. More concretely, consider a signal $\vec{I}\in\reals^{N}$ representing $N$ samples of $I(t)$ at $t=0, \Delta t, \ldots$, with elements indexed as $\vec{I}(0), \vec{I}(1),\ldots$. To calculate $\gn{2}(\tau)$ for $\tau\in\integers^{*}$:
%\begin{equation}
%\gn{2}(\tau) = \frac{\sum_{j=0}^{N-\tau}{\vec{I}(j)\vec{I}(j+\tau)}}
%                    {\sum_{j=0}^{N-\tau}{\vec{I}(j)}\sum_{j=0}^{N-\tau}{\vec{I}(j+\tau)}}
%\end{equation}
%Note that, for $\tau\rightarrow N$, the number of elements in the sum approaches 0. This represents the undersampled region of the correlation, and as such it is necessary to define the correlation window as significantly smaller than the sampled window in order to obtain a meaningful estimate of \gn{n}.
%
%This definition of a signal is useful for many measurements of some gross quantity which can be said to sample a non-trivial range of values in \reals{} or \integers. However, photon-counting produces a signal which is fundamentally binary (in $\integers_{2}$), indicating that either a photon has arrived, or none has. In principle it is possible to bin these photon arrivals to count the number of arrivals in some time interval and recover the vector-like signal discussed above, but such steps introduce a range of subtle artifacts related to the precise origin of time and definition of bin resolution. These problems are largely avoidable if we instead develop a definition of \gn{n} which involves counting these events directly.
%
%Do note, however, that the ``true'' photon arrival time discussed from here on is itself a binary form of this vectorial definition, because real instruments will have some finite timing resolution. In this sense, the idea of a discrete arrival time is just a simplification of the true signal, where each sampling represents the state of a photon arriving or not arriving during that interval. Many more samples will find no photon than one, so those are simply not reported. Additionally, this means that the reported photon arrival time carries some time units defined by the resolution of the measurement, so we can declare any arrival time $t$ to be a multiple of these time steps, or $t\in\wholes$, where \wholes{} is the set of all whole numbers (the positive integers and zero).  
%
%\subsection{Calculating \gn{n} by counting photons}
%As in equation~\ref{eq:delta_train}, it is possible to define the signal representing arrivals of photons as a sum over a set of $\delta$-functions. Consider the set $T\subset\wholes$ of photon arrival times. The signal can be defined as
%\begin{equation}
%I(t) = \sum_{t'\in T}{\delta(t-t')}
%\end{equation}
%This notation is cumbersome, so from here we will refer to a photon arrival time as $t'$ alone, but the $\delta$ notation could be substituted as desired. This change makes the summation notation difficult to parse, so we instead switch to a set notation:
%\begin{equation}
%I(t) = \abs{\braces{\left. t'\right|t'\in T;~t-t'=0}}
%\end{equation}
%In long form, this definition counts the number of photon arrival times $t'$ which are equal to the requested time $t$. This is computationally inefficient but conceptually simple, so we will define all important quantities in this fashion before discussing how to compute the result efficiently.
%
%Extending this notation to \gn{2} for a single signal:
%\begin{equation}
%\gn{2}(\tau) = \frac{\abs{\braces{(t_{j}, t_{k})\left|
%                          \begin{split} 
%                            t_{j}, t_{k}\in T; \\
%                            t_{j}-t_{k}=\tau
%                          \end{split}\right.
%                    }}}
%                    {\abs{\braces{T}}^{2}/\left(max(T)-min(T)\right)^{2}}
%\end{equation}
%where $min$ and $max$ are the functions which return the minimum and maximum values of a set, respectively. Even this definition is not quite sufficient: the measurement carries its own unit of time $\epsilon$, which means that any time $t$ specified is really a range $[t,t+\epsilon)$, so with appropriate normalization the result becomes:
%\begin{equation}
%\gn{2}(\tau) = \frac{\abs{\braces{(t_{j}, t_{k})\left|
%                          \begin{split} 
%                            t_{j}, t_{k}\in T \\
%                            t_{j}-t_{k}=\tau
%                          \end{split}\right.
%                    }}}
%                    {\epsilon^{2}\abs{\braces{T}}^{2}/\left(max(T)-t\right)^{2}}
%\end{equation}
%This result is identical to the normalization of histogrammed values, which will be discussed later.
%
%Extending this result to a number of signals, we obtain
%\begin{equation}
%\label{eq:gn_set}
%\gn{n}(\tau_{1}, \ldots) = \frac{\abs{\braces{(t_{0}, t_{1}, \ldots)\left|
%                                      \begin{split}
%                                      t_{0}\in T_{0}; t_{1}\in T_{1};\ldots \\
%                                      t_{1}-t_{0} = \tau_{1}; \ldots
%                                      \end{split}\right.}}}
%                                {\prod_{j=0}^{n-1}{\epsilon_{j}\frac{\abs{T_{j}}}{max(T_{j})-min(T_{j})}}}
%\end{equation}
%Typically, the signals being correlated will come from the same device, such that all $\epsilon_{j}$ are equal, leading to the final formula:
%\begin{equation}
%\label{eq:gn_set}
%\gn{n}(\tau_{1}, \ldots) = \frac{\abs{\braces{(t_{0}, t_{1}, \ldots)\left|
%                                      \begin{split}
%                                      t_{0}\in T_{0}; t_{1}\in T_{1};\ldots \\
%                                      t_{1}-t_{0} = \tau_{1}; \ldots
%                                      \end{split}\right.}}}
%                                {\prod_{j=0}^{n-1}{\epsilon\frac{\abs{T_{j}}}{max(T_{j})-min(T_{j})}}}
%\end{equation}
%
%\subsection{Subdividing the problem of calculating \gn{n}}
%The expression in equation~\ref{eq:gn_set} is somewhat intimidating, but we can divide its 

\section{Picoquant}
\subsection{Purpose}
This program decodes binary data from the Picoquant hardware and outputs the data in human-readable format. Currently, \picoquant{} supports most modes and versions of the Timeharp, Picoharp, and Hydraharp. The board and mode are detected automatically, and if the mode is not supported an error message will explain the details.

Modes and software versions supported:
\begin{itemize}
\item Timeharp: v20 (thd), v30 (thd, t3r), v50 (thd), v60 (thd, t3r)
\item Picoharp: v20 (phd, pt2, pt3)
\item Hydraharp: v10 (hhd, ht2, ht3)
\end{itemize}

\subsection{Command-line syntax}

\begin{verbatim}
Usage: picoquant [-r] [-v] [-i file_in] [-o file_out]
                 [-p print_every] [-n number] [-b] [-z] [-t]

        -i, --file-in: Input file. By default, this is stdin.
       -o, --file-out: Output file. By default, this is stdout.
         -n, --number: Number of entries to process (most 
                       pertinent for tttr modes). By default, 
                       processes all records.
    -p, --print-every: Print a status message after processing 
                       a specified number of entries. By default,
                       no status message is printed.
        -v, --verbose: Print debug-level information.
    -r, --header-only: Print header information, but no entries. 
                       Useful for debugging and checking file 
                       integrity.
     -b, --binary-out: Output a binary stream instead of ascii. 
                       Refer to the documentation for each mode 
                       for the details of each stream type. 
                       Generally, this will be identical to the
                       ascii mode in typing.
-z, --resolution-only: Print the time resolution of the 
                       measurement in picoseconds, then exit.
          -t, --to-t2: Convert a t3 file to a t2 file. This
                       assumes the sync channel is regular and
                       consistent over the whole run.
           -h, --help: Print this message.

        The file type and version will be 
        detected automatically from the file header.
\end{verbatim}

\subsubsection{Input}
The input is either binary stream of data from the timing hardware, or the name of a file containing that data. The details of each format are too varied and verbose to be summarized here, but are laid out in detail in section~\ref{sec:formats}.

\subsubsection{Output}
\label{sec:picoquant_output}
The output will either be written to \stdout{} or the file specified. By default, the stream will be represented as ascii text, but use of the \texttt{--binary-out} flag will skip the formatted print in favor of a raw binary stream. 

All records are processed in the order they are found in the data stream. This is typically time-ordered.

\paragraph{Interactive}
Each record is of the form:
\begin{verbatim}
struct {
    unsigned int channel;
    double left_edge;
    unsigned int counts;
};
\end{verbatim}
printed as:
\begin{verbatim}
channel number, left edge of time bin, counts \n
\end{verbatim}

\paragraph{T2}
Each record is of the form:
\begin{verbatim}
struct {
    unsigned int channel;
    long long int time;
};
\end{verbatim}
printed as:
\begin{verbatim}
channel number, time \n
\end{verbatim}

\paragraph{T3}
Each record is of the form:
\begin{verbatim}
struct {
	unsigned int channel;
	long long int pulse_number;
	int time;
};
\end{verbatim}
printed as:
\begin{verbatim}
channel number, pulse number, time \n
\end{verbatim}
	
\subsection{Examples of usage}
\subsubsection{Reading header information}
\begin{verbatim}
> picoquant --file-in data.phd --header-only
Ident = PicoHarp 300
FormatVersion = 2.0
CreatorName = PicoHarp Software
CreatorVersion = 2.3.0.0
FileTime = 14/05/11 17:55:48
Comment = Untitled
NumberOfCurves = 8
   ....
\end{verbatim}

Reading these values is often a good way to check the integrity of a file, and to make sure that the correct settings are used in later processing. The keywords used here are consistent with those used in the documentation for the file type as provided by Picoquant, so common values like measurement resolution may not be identical across versions and boards. 

\subsection{Obtaining the resolution of a measurement}
Resolution values must be multiples of 1\pico\second{} for most of the devices, but for the Timeharp they are integer divisions of 1\nano\second, leading to non-integer multiples of 1\pico\second. As such, all resolution values are presented as floats, even those which could be written as integers.
\begin{verbatim}
> picoquant --file-in data.phd --resolution-only
0,1.280000e+02
1,1.280000e+02
2,1.280000e+02
3,1.280000e+02
4,1.280000e+02
5,5.120000e+02
6,1.280000e+02
\end{verbatim}
Interactive mode allows for a large number of curves, so the resolution report gives the resolution for each curve, by index.
\begin{verbatim}
> picoquant --file-in data.pt2 --resolution-only
1.280000e+02
\end{verbatim}

\subsection{Reading interactive data}
\begin{verbatim}
> picoquant --file-in data.phd
0,0.000,0
0,0.128,0
0,0.256,0
0,0.384,0
0,0.512,0
0,0.640,0
0,0.768,0
0,0.896,0
0,1.024,0
0,1.152,0
...
\end{verbatim}

\subsection{Reading T2/T3 data}
\begin{verbatim}
> picoquant --file-in data.pt2
0,7128264
0,20957636
0,33684532
0,36576452
0,42146280
0,42251400
0,65787700
0,75149552
0,86537580
0,109288316
> picoquant --file-in data.pt3
1,103,47360
1,109,47616
1,115,85760
1,115,248832
1,213,55552
1,245,244992
1,254,49920
1,267,69888
1,268,122368
1,274,58624
\end{verbatim}

\subsection{Translating T3 data to T2 data}
\begin{verbatim}
> picoquant --file-in data.pt3 --to-t2
1,41248184
1,43648488
1,46086680
1,46249752
1,85257256
1,98246952
1,101651952
1,106872024
1,107324512
1,109660816
\end{verbatim}

\subsection{Mode- and hardware-specific information}
\label{sec:formats}

While there are many differences between the formats of the files generated by the different boards and modes, they all follow a common structure:
\begin{itemize}
\item a general header identifying the board type and software version
\item a board-specific header, identifying hardware and software configuration
\item a mode-specific header
\item data
\end{itemize}

As such, the process of streaming data can be broken down into the following steps:
\begin{enumerate}
\item Identify the board type (\texttt{picoquant.c})
\item Identify the software version used to generate the file (\texttt{hydraharp.c}, \texttt{picoharp.c}, \texttt{timeharp.c}).
\item Determine the collection mode used (\texttt{hydraharp/hh\_*.c}, $\ldots$).
\item  \begin{enumerate}
  \item If the run is specified as resolution-only or header-only, print the appropriate values.
  \item Otherwise, read through the remaining header information and print the data.
  \end{enumerate}
\end{enumerate}

In this implementation, while the code used to produce the data and headers is very similar between software versions for a given board, there are some small differences which make a general program difficult to write. As such, each version is hard-coded, and any changes to the overall structure of the program must be rolled out to all versions. Fortunately, many common tasks such as printing of data are centralized in \texttt{picoquant.c}, so changes to the output format only require modification of a single function.

In principle, the code can be collected into a nicer data-streaming object which masks the translation process and yields only the resulting data stream. This is probably the most convenient way to deal directly with developing custom tools for data processing, but for most purposes passing the data through pipes should be sufficient. If such an interface is desired, the low-level translation functions (found in \texttt{*/*\_v*.c}) should be sufficient when wrapped with higher-level logical routines like those used to determine the board identity and version. The function structure is uniform across all versions and boards, which should simplify the wrapping process.

The remainder of this section is devoted to a discussion of the details important to each board and measurement type. Most of this information can be found in the manuals included with the hardware, but there is a significant amount of information which is documented in more scattered locations, such as the sample data code. This summary includes the details vital to understanding how the raw data are actually translated into the general data streams, and how various design decisions affect the quality and precision of the result.

\subsubsection{A word about external markers}
Many of the timing boards have a feature which allows the insertion of an external timing pulse into the signal, for use in TTTR modes to designate a raster scan or other time-dependent behavior. These records are not handled directly by \picoquant, but instead a message is passed to \stderr{} indicating that such a record exists. If different behavior is desired (perhaps assignment of the marker to a non-existent channel), modify the function \texttt{external\_marker} in \texttt{picoquant.c}. This will cause problems with the other programs, which assume every record originates from the true signal stream, so such an alteration must be accompanied by code which appropriately splices the signal stream, for example by halting collection into a histogram and initiation of a new one.

\subsubsection{T3 timing carries units of histogram bins, not time}
In T3 modes--including T3R for the Timeharp--the pulse dimension carries units of pulse number, but the time is not strictly a time with fixed units. Instead, the specified resolution of the interactive mode is the unit of time, such that the precision of the measurement is affected directly by the choice of resolution. As a result, the T3 record more directly reflects the index of the bin the record would have fallen into in interactive mode. In \picoquant{} this value is converted to units of \pico\second{} implicitly, but do be aware of this distinction.

\subsubsection{Translation of T3 to T2 data}
In principle, if the sync source of a T3-mode experiment arrives with perfectly uniform spacing, the pulse number can be said to represent some constant amount of time, and a T2-like record recovered. To do this, pass the flag \texttt{--to-t2} for an input of T3 data, but consider the implications of how this translation is performed (see section~\ref{sec:t3_to_t2}). 

In \picoquant, the average rate of sync pulses is used to determine the time unit a pulse carries.

%For a frequency $f$ of the arrival of a sync source, pulse number $p$ and time delay $t$ on channel $c$ map as:
%\begin{equation}
%\left(c, p_{j}, t_{j}\right) \rightarrow \left(c_{j}, \frac{p_{j}}{f} + t_{j}\right)
%\end{equation}
%In practice, this mapping is limited in precision in a few important ways. First, the period $1/f$ is not necessarily an integer multiple of 1\pico\second, and any precision will be lost beyond the decimal. This is not usually a problem because typical sync sources operate at 10\mega\hertz{} at most, for a precision of timing to one part in ten thousand, and a purity of the pulse train to this precision is not typical. Additionally, the timing jitter of most lasers will be well above this threshold, so this is not too great a concern.
%
%A more important concern is that of missed pulses. The T3 data reports the detected pulse number, and if a sync pulse is missed no record will exist. However, translation of any subsequent record to an absolute time will be incorrect by the period $1/f$, confusing matters where such spacing is important. Again, a well-designed system should not have problems beating this limit, but it is the greatest single source of error during the translation.
%
%Collectively, these errors can be expressed as:
%\begin{equation}
%\left(c_{j}, p_{j}+\delta p_{j}, t_{j} + \delta t_{j}\right) \rightarrow \left(c_{j}, \frac{p_{j}+\delta p_{j}}{f+\delta f} + \left(t_{j}+\delta t_{j}\right)\right)
%\end{equation}

\subsubsection{Timeharp}
The Timeharp is the least sophisticated of the three boards discussed here, and is fundamentally a histogramming board. It features two input channels, and its three modes are:
\begin{itemize}
\item interactive (thd): Collection of a histogram, using channel 0 as the sync and channel 1 as the signal.
\item continuous interactive (thc): Identical to normal interactive mode, except that the histogram is reported at the end of a user-specified time interval, repeating until halted. This feature is used for experiments requiring time resolution of a time-dependent feature, such as a fluctuating fluorescence lifetime. This feature is not supported by the existing software, due to lack of data or interest in using this mode (it is not present in the other timing boards).
\item time-tagged time-resolved (t3r): TTTR mode, equivalent to T3 mode, with channel 0 as the sync and channel 1 as the signal. Instead of reporting the delay time directly, this mode reports the index of the histogram bin an event would fall into. The time delay represented by the delay can be recovered from the header information, but for homogeneity of the treatment of T3-like data this step is not performed automatically.
\end{itemize}

\subsubsection{Picoharp}
The Picoharp features two input channels. Its three modes are:
\begin{itemize}
\item interactive (phd): Standard interactive mode, using channel 0 as the sync and channel 1 as the signal.
\item t2 (pt2): A T2 mode, with both channels treated equally and all pulse arrivals recorded. 
\item t3 (pt3): A T3 mdoe, with channel 0 as the sync and channel 1 as the signal.
\end{itemize}
In both the T2 and T3 mode, internal clocks of limited precision ($<$32 bits) are used to record the passage of time or pulses. As such, many records are devoted to recording the occurrence of an integer overflow. These records are treated by \picoquant{} automatically, producing data streams of the form specified in section~\ref{sec:picoquant_output}.

\subsubsection{Hydraharp}
The Hydraharp features a dedicated sync channel and four input channels. Its three modes are:
\begin{itemize}
\item interactive (hhd): Standard interactive mode, with four separate histograms assigned to  the four input channels.
\item t2 (ht2): A T2 mode, with all channels treated equally, including the sync channel. If the sync channel is active, these events will be recorded, and for convenience \picoquant{} will output each event as arriving on channel 4. This channel index can be altered by modifying value of the global variable \texttt{HH\_SYNC\_CHANNEL} (\texttt{hydraharp.h}) at compilation time. 
\item t3 (ht3): A standard t3 mode.
\end{itemize}
Future versions of the Hydraharp are being developed with extra input channels. If your model features extra channels, it should be sufficient to modify the value of \texttt{HH\_SYNC\_CHANNEL}, as the number of channels and modules is handled dynamically by \picoquant.

\section{Intensity}
\subsection{Purpose}
For T2 and T3 data, it is often useful to group records into fixed time bins and count the number of records in each bin. This program does just that, reporting the intensity at each channel for all data in a time-ordered stream. 

\subsection{Command-line syntax}
\begin{verbatim}
Usage: intensity [-v] [-i file_in] [-o file_out] [-c channels]
                 -w bin_width -m mode

           -v, --verbose: Print debug-level information.
           -i, --file-in: Input file. By default, this is
                          STDIN.
          -o, --file-out: Output file. By default, this is
                          STDOUT.
         -w, --bin-width: Width of the bin, in time or pulses
                          depending on the mode.
              -m, --mode: Stream type. This is either t2 or t3,
                          and the style of the output will be
                          different for each.
          -c, --channels: Number of channels in the stream. By
                          default, this is 2.
         -a, --count-all: Rather than counting records in 
                          distinct time bins, count all records 
                          in the stream.
              -h, --help: Print this message.

       This program assumes the input stream is time-ordered.
       
       If counting all events or printing the last one, the time
       written will be the time of the last event seen.
\end{verbatim}

\subsubsection{Input}
T2 and T3 modes accept data of the form produced by \picoquant, as specified in section~\ref{sec:picoquant_output}.

\subsubsection{Output}
The output for both T2 and T3 modes is of the form:
\begin{verbatim}
lower time limit, upper time limit, 
  channel 0 intensity, channel 1 intensity, ... \n
\end{verbatim}
For T2 mode, the time bin is defined as a number of picoseconds. For T3 mode, the time bin is a number of pulses.

Passing the flag \texttt{--count-all} will ignore the time bins, and instead count all events on each channel as belonging to one large channel. 

\subsection{Examples of usage}
Count into 10\milli\second{} bins.
\begin{verbatim}
> picoquant --file-in data.pt2 --number 100000 |  \
  intensity --bin-width 10000000000 --mode t2 --channels 2 
0,10000000000,555,0
10000000000,20000000000,524,0
20000000000,30000000000,564,0
30000000000,40000000000,536,0
40000000000,50000000000,495,0
50000000000,60000000000,530,0
60000000000,70000000000,514,0
70000000000,80000000000,548,0
80000000000,90000000000,558,0
90000000000,100000000000,565,0
100000000000,110000000000,550,0
110000000000,120000000000,490,0
120000000000,130000000000,559,0
130000000000,140000000000,504,0
140000000000,150000000000,547,0
150000000000,160000000000,537,0
160000000000,170000000000,535,0
170000000000,180000000000,545,0
180000000000,186537236212,344,0
\end{verbatim}
Count all events:
\begin{verbatim}
> picoquant --file-in data.pt2 --number 100000 |  \
  intensity --bin-width 10000000000 --mode t2 --channels 2 \
  --count-all
0,186537236212,10000,0
\end{verbatim}

\subsection{Implementation details}
\label{sec:intensity_implementation}
Each channel can be treated independently, so we can focus on how to handle a single stream of records. 

Given a stream of events with associated arrival times $t_{k}$ and time-ordered time bins with boundaries defined as $[b_{j}, b_{j+1})$, the intensity $I(j)$ of the stream in a given time bin is defined as:
\begin{equation}
I(j) = \left| \braces{t_{k} | t_{k} \in \left[b_{j}, b_{j+1}\right)} \right|
\end{equation}

For a disordered stream of length $N$ and $M$ time bins, it is most efficient to place each entry into a bin in the order they arrive, which means that the true intensity of the signal cannot be known until all elements are processed. Because the bin assignment is continuous and unambiguous, a binary search algorithm can be used to determine the correct bin, in O($\log{(\textnormal{M})}$) for an element. If the spacings are linear, this problem is O(1). Thus, the whole stream can be processed in O(N) time, or O($\textnormal{N}\log{(\textnormal{M})}$) time for the non-linear case. In principle, the disordered stream could be ordered at a cost of O($\textnormal{N}\log{(\textnormal{N})}$), but this extra factor makes the process too expensive for most purposes.

If instead the stream is time-ordered, as is the case for data from \picoquant, it is more efficient to choose a time bin and count the associated elements. In this case, an entry is drawn from the stream, and if it falls in the bin the count for that bin is incremented. If the entry does not fall in the stream, the stream has passed that bin and the count for that bin is reported. In this latter case, the element from the stream is retained until its bin is found. This process is overall of order O(N), because there is only a constant cost associated with moving between the bins, even if they are not regularly spaced.

The algorithm can be expressed more concisely as:
\begin{enumerate}
\item Set $b$ to the next bin bounds, the bin intensity $I\leftarrow 0$, and draw a time $t$ from the stream.
\item If either $t$ or $b$ does not exist, go to 5. 
\item If $t\in b$, increment $I$, draw a new time $t$ from the stream, and go to 2.
\item We are outside the bin now. Yield $(b,I)$, set $I\leftarrow 0$, draw a new bin $b$, and go to 2.
\item We have reached the end of the stream. Yield $(b,I)$ and halt.
\end{enumerate}
%\begin{verbatim}
%bin = next(bins)
%intensity = 0
%time = next(data_stream)
%
%while time != None and bin != None:
%    if time in bin:
%        intensity += 1
%        time = next(data_stream)
%    else:
%        yield(bin, intensity)        
%        intensity = 0
%        bin = next(bins)
%
%yield(bin, intensity)
%\end{verbatim}
As is evident from this algorithm, the process can be run continuously for a stream, yielding results without the need to process the entire stream. In practice, the final step will yield the lower bound of the bin and the final time seen, to avoid undersampling issues.

\section{Correlate}
\label{sec:correlate}
The details of the mathematics required for understanding this program can be found in section~\ref{sec:math_background}. If you have not read that section, the discussion of the implementation here will be difficult to follow, but a brief explanation of the background will be given at each step.
%\subsection{A brief mathematical background}
%\label{sec:correlate_math}
%For several types of experiments, some form of a signal correlation is necessary. These follow the form:
%\begin{equation}
%\label{eq:correlation}
%\gn{n}(\tau_{1}, \ldots \tau_{n-1}) = \frac
%	{\angles{I_{0}(t)\prod_{j=1}^{n-1}{I_{j}(t+\tau_{j})}}}
%	{\angles{I_{0}(t)}\prod_{j=1}^{n-1}{\angles{I_{j}(t+\tau_{j})}}}
%\end{equation}
%where $I_{j}(t)$ are the intensities of some number of signals and the angled brackets indicate an average over time. For some purposes, the autocorrelation of a signal is of interest, where all $I_{j}(t)$ are identical, but in other cases some mixture of signals is of interest, and not necessarily in numerical order. For example, fluoroescence correlation spectroscopy requires the calculation of a second-order autocorrelation of fluorescence intensity, or:
%\begin{equation}
%\label{eq:g2}
%\gn{2}(\tau) = \frac{\angles{I(t)I(t+\tau)}}
%                 {\angles{(I(t)}\angles{(I(t+\tau)}}
%             = \frac{\angles{I(t)I(t+\tau)}}
%                 {\angles{(I(t)}^{2}}
%\end{equation}
%where $I(t)$ includes signal across all detection channels. In most cases, calculation of \gn{2} is sufficient, but in this definition we have implicitly limited our discussion to T2-like signals, with one independent variable per signal. In the case of T3-mode signals, we really have two independent variables: pulse number and arrival time. In principle, we could write something like
%\begin{equation}
%\gn{n}(\rho_{1}, \tau_{1}, \ldots \rho_{n-1}, \tau_{n-1}) = \frac
%	{\angles{I_{0}(p, t)\prod_{j=1}^{n-1}{I_{j}(p+\rho_{j},t)I_{j}(p,t+\tau_{j})}}}
%	{\angles{I_{0}(p, t)}^{2}\prod_{j=1}^{n-1}{\angles{I_{j}(p+\rho_{j},t)}
%	                                       \angles{I_{j}(p,t+\tau_{j})}}}
%\end{equation}
%where the average is now over all pulse and real time. However, because both pulse and real time are integer spaces, we can just map the times onto two homogeneous dimensions, and perform the correlation \gn{2n} instead. This gives a \gn{n} for every choice of pulses, useful for comparisons of long- and short-time correlations, as in multi-excition spectroscopy.

\subsection{Purpose}
This program calculates the full correlation of a signal, including all autocorrelations and cross-correlations of individual channels, for arbitrary numbers of channels and order. The raw correlation events are returned as the output, \textit{not the histogrammed correlation}. The histogramming and normalization are left for other programs, such as \texttt{histogram}.

The speed of the calculation scales favorably with the order of correlation, although of course higher orders of correlation require greater numbers of events to build meaningful results. See section~\ref{sec:correlation_implementation} for details of the algorithm.

\subsection{Command-line syntax}
\begin{verbatim}
Usage: correlate [-v] [-i file_in] [-o file_out] [-a] [-b] 
                 [-n number] [-p print_every] [-q queue_size]
                 [-d max_time_distance] [-e max_pulse_distance]
                 [-r] -g order -c channels -m mode

           -v, --verbose: Print debug-level information.
           -i, --file-in: Input file. By default, this is 
                          STDIN.
          -o, --file-out: Output file. By default, this is 
                          STDOUT.
       -p, --print-every: Print the result for multiples of 
                          this number of entries. Default is to
                          print nothing.
              -m, --mode: Stream type. This is either t2 or t3,
                          and the style of the output will be
                          different for each.
        -q, --queue-size: Defines the maximum length of the 
                          circular queue held in memory for
                          processing. By default, this is 100000
 -d, --max-time-distance: Defines the maximum difference in time
                          that two entries can have and still be
                          considered for correlation (t2 and t3).
-e, --max-pulse-distance: Defines the maximum difference in pulse
                          number that two entries can have and 
                          still be considered for correlation 
                          (t3 only).
             -g, --order: Order (g(n)(t1...tn-1)) of the 
             			  correlation to perform. By default this
             			  is 2, the standard cross-correlation of
             			  two channels.
          -c, --channels: Number of channels in the incoming
                          stream. By default, this is 2 
                          (Picoharp).
  -r, --channels-ordered: Organize the output such that the 
                          channels are in order. By default, 
                          this is not performed.
              -h, --help: Print this message.

       This program assumes the input stream is time-ordered.
\end{verbatim}

\subsubsection{Input}
T2 and T3 modes accept data of the form produced by \picoquant, as specified in section~\ref{sec:picoquant_output}.

\subsubsection{Output}
The exact output will depend on the mode and order of correlation, but it always adheres the following form:
\begin{verbatim}
channel 0, channel 1, time 1, ..., channel 2, ... \n
\end{verbatim}
where the times are either times or pulses, and follow the order (pulse, time). Also, the channels here are not necessarily ordered, because the exact order of the signals in the correlation can be of great importance. For example, for a \gn{2} of T2 data:
\begin{verbatim}
channel 0, channel 1, time 1-time 0 \n
\end{verbatim}
for \gn{3} of T2 data:
\begin{verbatim}
channel 0, channel 1, time 1-time 0, 
  channel 2, time 2-time 0 \n
\end{verbatim}
for \gn{3} of T3 data:
\begin{verbatim}
channel 0, channel 1, pulse 1-pulse 0, time 1-time 0,
   channel 2, pulse 2-pulse 0, time 2-time 0 \n
\end{verbatim}

By default, the ordering of the channels is that found in the stream. However, for some applications it is useful to order the channels and record both positive- and negative-time correlations on the same histogram. To account for this, pass the flag \texttt{--channels-ordered} to order the channels and apply the appropriate sign to the time differences.

By default, the program will correlate all entries in a stream. This can easily cause memory problems, so it is recommended that a maximum time distance is specified for T2 data, or a maximum pulse distance for T3 data. A maximum time distance may also be specified for T3 data, but in principle this is unnecessary. Note that all channels are treated equally by these limits. 

Currently, \correlate{} uses a fixed-length circular buffer to store entries, so if errors report that the buffer is too small try changing the length with \texttt{--queue-size}.

\subsection{Examples of usage}
Finding the correlation events for a \gn{2} of T2 data:
\begin{verbatim}
> picoquant --file-in data.ht2 --number 10000 | \
  correlate --channels 4 --mode t2 \
  --max-time-distance 1000
3,2,932
3,0,558
3,1,508
\end{verbatim}
the same, with channels ordered:
\begin{verbatim}
> picoquant --file-in data.ht2 --number 10000 | \
  correlate --channels 4 --mode t2 \
  --max-time-distance 1000 --channels-ordered
2,3,-932
0,3,-558
1,3,-508
\end{verbatim}
\gn{2} of T3 data:
\begin{verbatim}
> picoquant --file-in data.ht3 --number 100 | \
  correlate --channels 4 --mode t3 \
  --max-pulse-distance 1000
3,3,447,14436
1,1,115,-15636
3,2,369,-28240
1,1,986,66088
0,3,240,21380
\end{verbatim}
\gn{3} of T3 data:
\begin{verbatim}
> picoquant --file-in data.ht3 --number 1000 | \
  correlate --channels 4 --mode t3 --order 2 \
  --max-pulse-distance 1000
3,3,10,-44540,0,913,8648
3,0,155,9044,2,353,-792
2,3,128,47316,3,144,42004
2,2,44,-32120,3,300,1380
2,2,44,-32120,2,359,-4924
2,3,300,1380,2,359,-4924
2,3,256,33500,2,315,27196
1,3,460,72356,2,847,20448
3,2,387,-51908,3,603,-43404
2,1,78,-33380,2,323,-60748
1,3,54,-39056,2,239,-60648
3,2,139,39596,3,364,11248
2,2,27,-82228,3,999,-10108
\end{verbatim}

\subsection{Implementation details}
\label{sec:correlation_implementation}
The math and notation from section~\ref{sec:math_background} will feature heavily in this section, so you should become familiar with the results of that section before proceeding. 

The problem \correlate{} must address is that of generating efficiently the events which occupy one of the sets in the sum in equation~\ref{eq:gn_set}. That is, we must find all events of the form
\begin{equation}
\braces{(\photon_{0},\ldots\photon_{n-1})
        \left|
        \begin{aligned}
        \photon_{0}\in\photons_{c_{0}};\\
        \ldots; \\
        \Time(\photon_{1})-\Time(\photon_{0})=\tau_{0};\\
        \ldots;
        \end{aligned}
        \right.}
\end{equation}
In long form, we must find all tuples of photons which satisfy specific rules for their relative time delays. To do this, we will start with the information we can know or enforce for the incoming stream of photons \photons:
\begin{enumerate}
\item The set \photons{} of all photons is time-ordered ($\Time(\photon_{j})\le\Time(\photon_{k})$ for $j<k$).
\item The correlation will only be meaningful for a time window $\timewindow\subset\wholes$ much smaller than the integration time window ($[0, \integrationtime)$).
\end{enumerate}
This latter rule can be expressed as a modification of the conditions in the set:
\begin{align}
\photons_{\timewindow}\equiv\braces{\photon\left|\photon\in\photons;~\Time(\photon)\in\timewindow\right.}\\
\braces{(\photon_{0},\ldots\photon_{n-1})
        \left|
        \begin{aligned}
        \photon_{0}\in\photons_{c_{0}}\cap\photons_{\timewindow};\\
        \ldots; \\
        \Time(\photon_{1})-\Time(\photon_{0})=\tau_{0};\\
        \ldots;
        \end{aligned}
        \right.}
\end{align}
The former rule simplifies the algorithm we will devise to perform this calculation.

\subsubsection{The photons in a time window can be found efficiently}
Without loss of generality, we define a window window of time $\timewindow$ such that, for some bounds $a,b\in\wholes$, $a<b$:
\begin{equation}
\timewindow = [a,b)
\end{equation}
If we allow $a\rightarrow0$ and $b\rightarrow\integrationtime$, we recover the full set of times for the experiment, but for practical matters we will focus on this window \timewindow.

Given that \photons{} is time-ordered, we can always draw from it a photon \photon{} with minimal time, so from here we will treat \photons{} as a queue, a structure which holds an arbitrary number of elements in a well-defined order. This queue is time-ordered, although there is some ambiguity in how to order photons arriving on different channels at the same time. Ultimately, the ordering of such photons is not important, but we can define a second order parameter for the channel if needed.

In principle we could enumerate all time windows \timewindow{} with some specified \abs{\timewindow} and find events in that window, but there are far more windows than photons (otherwise, we could have treated the signal as a vector, as in section~\ref{sec:sampling_intensity}). Therefore, it is much more efficient to choose each window as starting at some photon, and draw photons from the queue until the window limit is surpassed. Once we have drawn this sub-queue $\photons_{\timewindow}$ we can turn our attention to generation of photon tuples from that set.

Up to the tuple generation, our algorithm is:
\begin{enumerate}
\item[0.] Set $\photons_{\timewindow}=\emptyset$, correlation order $n$.
\item Draw the next element $\photon$ from $\photons$:
  \begin{enumerate}
  \item If $\photon$ exists, add it to the last position in $\photons_{\timewindow}$.
  \item If no $\photon$ exists (the queue is empty):
    \begin{enumerate}
    \item If $\abs{\photons_{\timewindow}}\ge n$, generate tuples from $\photons_{\timewindow}$.
    \item Halt.
    \end{enumerate}
  \end{enumerate}
\item Call $\photon_{-1}$ the last photon in $\photons_{\timewindow}$ and $\photon_{0}$ the first.
\item If $\Time(\photon_{-1})-\Time(\photon_{0})\ge\abs{\timewindow}$, go to 1.
\item If $\abs{\photons_{\timewindow}}\ge n$, generate tuples from $\photons_{\timewindow}$.
\item Remove the $\photon_{0}$ from $\photons_{\timewindow}$, then go to 1.
\end{enumerate}
From this algorithm it is evident that the cost of building the windows scales most directly with the number of photons (we never have to perform more than $\abs{\photons}$ iterations of this algorithm), giving O(\abs{\photons}) for this step. 

To see the implementation of this algorithm, the relevant files are \texttt{correlate.c}, \texttt{correlate\_t2.c}, and \texttt{correlate\_t3.c}. In particular, the functions \texttt{*queue*} are the most relevant to this section, as these handle the population and depopulation of \photons{} and $\photons_{\timewindow}$.

\subsubsection{Generating correlation events (photon tuples) from $\photons_{\timewindow}$}
Given the algorithm to find all subqueues $\photons_{\timewindow}$, we turn our attention to finding the tuples $(\photon_{0},\ldots\photon_{n-1})$ in a given $\photons_{\timewindow}$. Formally, we want to generate the elements of the set
\begin{equation}
\braces{(\photon_{0},\ldots\photon_{n-1})\left|\photon_{0},\ldots\in\photons_{\timewindow}\right.}
\end{equation}
This is actually a quite familiar problem, if we ignore the photon structure and instead focus on producing the appropriate indices from the queue. Call $N\equiv\abs{\photons_{\timewindow}}$, and note that we are looking to produce all unique permutations of $n$ elements of $\integers_{N}$. These can be ordered to produce:
\begin{align*}
&(0, 0,\ldots 0, 0) \\
&(0, 0, \ldots 0, 1) \\
&\ldots \\ 
&(0, 0, \ldots 0, N-1) \\
&(0, 0, \ldots 1, 0) \\
&\ldots 
\end{align*}
These can be enumerated as the numbers $0,1,\ldots N^{n}-1$ in base $N$, which can be seen by treating each element of the tuple as a coefficient in the base expansion of a number. For example, any number $N\in\wholes$ can be expressed as a sum over powers of 2:
\begin{equation}
N = \sum_{j=0}^{n}{c_{j}2^{j}}
\end{equation}
for some $n\in\wholes$. While this is true, it is also subject to some redundancy, because we have shown that the positive-time correlation is sufficient for reconstruction all negative times (see section~\ref{sec:correlation_function}), so we actually only need to compute this factor for all time-ordered tuples. Additionally, any tuple containing the same photon twice is not important, because there will never be any structure to a photon correlated with itself. Thus we need only include the tuples for which the indices are ordered and unique, the upper hypertriangle in index space.

Additionally, because the window will move to eliminate the first photon, any correlation not involving this photon should not be handled now. Thus the index tuples of interest are:
\begin{align*}
&(0,1,\ldots,n-2, n-1)\\
&(0,1,\ldots,n-2, n)\\
&\ldots\\
&(0,1,\ldots,n-2,N-1)\\
&(0,1,\ldots,n-1,n)\\
&\ldots\\
&(0,N-n,\ldots N-1)
\end{align*}

To generate these combinations of indices, apply the following algorithm:
\begin{enumerate}
\item[0.] Define $x\leftarrow(0,1,\ldots n-1)$, and index the $j$th element of $x$ as $x_{j}$. 
\item Yield $x$.
\item Increment as many digits of $x$ as necessary:
  \begin{enumerate}
  \item Set $j\leftarrow n-1$.  
  \item If $j\le 1$, halt this loop.
  \item Increment $x_{j}$. 
  \item If $x_{j}\ge N$, set $x_{j}\leftarrow 0$, decrement $j$, and go to 1(b) (overflow of this digit).
  \item Otherwise, set $j\leftarrow 0$, decrement $j$, and go to 1(b) (no overflow, stop incrementing).
  \end{enumerate}
\item Refill the overflowed digits of $x$:
 \begin{enumerate}
 \item If $x_{1}=0$, we have incremented up to the first index. Set $x_{1}\leftarrow x_{1}'+1$.
 \item Set $j\leftarrow 1$.
 \item If $j\ge n$, halt this loop.
 \item If $x_{j}=0$, set $x_{j}\leftarrow x_{j-1}+1$ (increment the digit based on the previous non-overflowed digit).
 \item Increment $j$, and go to 3(c).
 \end{enumerate}
\item If $x_{n-1}\ge N$, halt. This is not a valid tuple of indices, which means we have exhausted all of the valid tuples.
\item Otherwise, go to 1.
\end{enumerate}
This algorithm is not perfect, as we must loop over all elements of the tuple in the refilling step. This is ultimately a minor cost and could be accounted for in the implementation, but for \correlate{} the algorithm was implemented as described.

In long form, we start at the right end of the tuple, incrementing values while moving left until we find a value which does not overflow past $N$. Once we find this value, we stop incrementing, then form the next possible tuple from the leading non-zero index, or the previous value of the leading digit if we have overflowed that as well. If the tuple can be refilled without overflowing, it is valid and we yield it. Otherwise, we have reached the end and are finished.

From the algorithm, we see that an increment costs, at worst, O($n$) loops. In reality the average is somewhat lower than this, and could be calculated exactly with some effort. Additionally, this algorithm will produce some number of tuples of somewhat less than order O($N^{n-1}$). This behavior is sensible: a larger window ($N=\abs{\photons_{\timewindow}}$) gives more valid tuples of photons, and the higher order $n$ gives more iterations over that window.

To see the implementation of this algorithm in \correlate, examine the functions \texttt{*offsets*} in \texttt{combinations.c}.

\subsubsection{Production of the correlation event, given a photon tuple}
There are two distinct modes enabled in \correlate{} for producing the correlation events: with and without ordering of channels. We will deal with the time-ordered, channel-unordered case first. 

Given a tuple of photons $(\photon_{0},\ldots\photon_{n-1})$ known to satisfy the conditions for including in calculating \gn{n}, we must produce the time differences $(\tau_{1}, \ldots\tau_{n-1})$. This can be done with the following algorithm:
\begin{enumerate}
\item[0.] Set $j\leftarrow 1$.
\item If $j\ge n$, halt and yield $(\tau_{1},\ldots\tau_{n-1})$.
\item Set $\tau_{j}\leftarrow \Time(\photon_{j}) - \Time(\photon_{0})$.
\item Increment $j$ and go to 1.
\end{enumerate}
This process scales as O($n$), although the constant factor is quite small relative to the other steps in the overall algorithm. Correlation for T3 mode involves a second term for the pulse difference, but it otherwise identical.

To perform this same step for channel-ordered (time-unordered) photons, we can pre-populate a list of all possible channel orders and combinations and the order of the indices to choose, instead of incrementing from start to finish. The details of this will be described in section~\ref{sec:histogram}, but the result is that the cost of this operation is only upfront, and there is essentially no additional cost once the list of combinations exists.

%\subsubsection{Definition of the correlation as the order of a set}
%As mentioned in section~\ref{sec:correlate_math}, T3 and T2 data are closely related and can be treated similarly. To understand how this is possible, it is worth spending some time considering what information is required to compute the correlation.
%
%In equation~\ref{eq:g2}, the denominator is an average intensity of a signal. This can be computed simply by determining the duration of the signal and the number of counts over that interval, and can be handled by using the result of \intensity{} \texttt{--count-all}. Thus the real problem is the computation of the numerator, which is somewhat imposing at first glance:
%\begin{equation}
%\angles{I_{0}(t)\prod_{j=1}^{n-1}{I_{j}(t+\tau_{j})}}
%\end{equation}
%To simplify matters, we will start with the autocorrelation of a signal channel:
%\begin{equation}
%\angles{I(t)I(t+\tau)}
%\end{equation}
%In this situation, the signal can be thought of as being composed of some sum of delta functions with peak centers at the times photons arrived. Thus the contribution of any given pair of photons to the correlation at time $\tau$ is nonzero only if the difference of their arrival times is $\tau$. Thus, for a given value of $\tau$ and photon arrival times \braces{t}:
%\begin{equation}
%\gn{2}(\tau) \propto \left| \braces{(t_{j}, t_{k}) | t_{j},t_{k}\in\braces{t}; t_{j}-t_{k}=\tau} \right|
%\end{equation}
%This result can be extended to higher orders by enforcing the restriction that a tuple of time differences must be satisfied:
%\begin{equation}
%\begin{split}
%\gn{n}(\tau_{1}, &\ldots \tau_{n-1}) \propto \\
%   &\left| \braces{
%       (t_{0}, \ldots t_{n-1})
%       \left|\begin{split}
%       t_{0},\ldots t_{n-1}\in\braces{t}; \\
%       (t_{1}, \ldots t_{n-1}) - (t_{0}, \ldots t_{0}) = (\tau_{1}, \ldots \tau_{n-1})
%       \end{split}\right.
%%       \begin{gather*}
%%        \\
%%       
%%       \end{gather*}
%   } \right|
%\end{split}
%\end{equation}
%
%\subsubsection{Mapping T3 data onto T2-like data}
%Now that we have a general expression for calculation of \gn{n}, it is worthwhile to make an aside describing how to map T3 data onto T2-like data for correlation. Consider the basic form of a tuple of $n$ T3 records:
%\begin{equation}
%\left((c_{0}, p_{0}, t_{0}), \ldots (c_{n-1}, p_{n-1}, t_{n-1})\right)
%\end{equation}
%which can be mapped isomorphically onto:
%\begin{equation}
%\left((c_{0}, p_{0}), (c_{0}, t_{0}), \ldots (c_{n-1}, p_{n-1}), (c_{n-1}, t_{n-1})\right)
%\end{equation}
%which looks like a tuple for T2 records:
%\begin{equation}
%\left((c_{0}, t_{0}), \ldots (c_{n-1}, t_{n-1})\right)
%\end{equation}
%Ultimately, a T3 tuple of length $n$ can be mapped onto a corresponding tuple of length $2n$, such that any correlation \gn{n} of T3 data can be treated exactly as a corresponding correlation \gn{2n} of T2 data.
%
%\subsubsection{Correlation of the time-ordered stream}
%As with \intensity, \correlate{} expects a time-ordered stream (see section~\ref{sec:intensity_implementation}). 
%
%% For a correlation of order $n$, every unique combination of $n$ elements of the stream can contribute to \gn{n}, so the full correlation of stream of lengh $N$  costs O(N$^{n}$) to compute. This is reduced for a fixed window width to approximately O(Nw$^{n-1}$) for a window of width w, for reasons that will become clear shortly.




\section{Histogram}
\label{sec:histogram}

\subsection{Purpose}
This program performs a multi-dimensional histogram in time and pulse dimensions for T2 and T3 correlation events produced by \correlate. The output is a set of histogram bins and the number of events which fell into each. This output is \textit{not} normalized by bin width or any other factor, and represents the raw number of counts falling into each bin.

Additionally, \histogram{} can be used to build a histogram of T3 events as would be done in interactive mode. To do this, set the mode to T3, and order to 1.

For \gn{n>2}, all time dimensions are defined identically, and all pulse dimensions are defined identically.


\subsection{Command-line syntax}
\begin{verbatim}
Usage: histogram [-v] [-i file_in] [-o file_out] [-p print_every]
                 -d <left_time_limit, time_increment, 
                     right_time_limit> 
                 -e <left_pulse_limit, pulse_increment, 
                     right_pulse_limit> 
                 -c channels -g order -m mode

         -v, --verbose: Print debug-level information.
         -i, --file-in: Input file. By default, this is stdin
        -o, --file-out: Output file. By default, this is stdout.
            -m, --mode: Stream type. This is either t2 or t3, 
                        and the style of the output will be 
                        different for each.
     -d <time>, --time: The upper and lower bounds for the time
                        axis in the histogram, along with the 
                        number of bins to create. The required 
                        format is lower,bins,upper (no spaces).
   -e <pulse>, --pulse: Same as time, but for pulses. This is 
                        only relevant in t3 mode.
        -c, --channels: The number of channels in the incoming
                        stream. By default, this is 2 (Picoharp
                        or TimeHarp).
           -g, --order: The order of the correlation performed.
                        By default, this is 2 (the standard 
                        cross-correlation.
      -D, --time-scale: Sets whether the time scale is "linear", 
                        "log". or "log-zero" (includes zero-time
                        peak in the first bin. The default is a
                        linear scale.
     -E, --pulse-scale: Sets whether the pulse scale is "linear", 
                        "log", or "log-zero". The default is a 
                        linear scale.
            -h, --help: Print this message.

            This program assumes that the channels are presented
            in order.
\end{verbatim}

\subsubsection{Input}
The expected input is the ascii output of \correlate, except for an order 1 of T3 data. These parameters specify a histogramming of the individual timing events, to recover the histogram which would have been obtained in interactive mode.

As for the rest of the parameters, most of these follow the same pattern as \correlate and therefore will not be discussed here. The new flags involve definitions for the time and pulse dimensions: \texttt{--time}, \texttt{--time-scale}, \texttt{--pulse},  and \texttt{--pulse-scale}. The time and pulse dimensions behave identically, so we will only discuss the properties of one here.

For \texttt{--time}, the parameters specify the lower and upper bounds of the time axis, as well as the number of bins $n$ to create along that axis. For a linear spacing, the bin width $\Delta t$ is $(t_{\max}-t_{\min})/n$, such that the bins are defined by the ranges
\begin{equation}
\begin{aligned}
&[t_{\min},t_{\min}+\Delta t),\\
&[t_{\min}+\Delta t,t_{\min}+2\Delta t),\\
&\ldots\\
&[t_{\min}+(n-1)\Delta t,t_{\min}+n\Delta t)
\end{aligned}
\end{equation}
This linear spacing is the default behavior, but the flag \texttt{--time-scale} can produce two other scales: log, and log-zero. The log scale creates bins with fixed width over the span of $[\log(t_{\min}),\log(t_{\max}))$, as if often desired for measurements requiring long and short time correlations. The log-zero scale has identical behavior, except that any zero-time correlations ($\tau=0$) are placed into the first bin. Note that the log scale cannot handle zero-time correlations, and neither log not log-zero can handle negative-time correlations. These values will be dropped from the histogram, with an error message indicating this has happened.

\subsubsection{Output}
After the input stream terminates, \histogram{} outputs the bin definitions and the number of counts associated with that bin. Generically, this format is:
\begin{verbatim}
channel 0, channel 1, bin (1,1) lower, bin (1,1) upper, ...,
    channel 2, ... , 
    counts \n
\end{verbatim}
where the channels are integers, bin edges are floats, and the counts are integers. For every bin in the histogram, one line will be output. For T2 mode, the bin definition has only one dimension (time), so the format is:
\begin{verbatim}
channel 0, channel 1, time 1 lower, time 1 upper, 
           channel 2, time 2 lower, ...,
           counts \n
\end{verbatim}
T3 data have an additional dimension (pulse):
\begin{verbatim}
channel 0, channel 1, pulse 1 lower, pulse 1 upper,
                      time 1 lower, time 1 uppper,
           channel 2, ...
           counts \n
\end{verbatim}
See section~\ref{sec:histogram_examples} for specific examples of output in these formats.


\subsection{Examples of usage}
\label{sec:histogram_examples}
\subsubsection{Time-averaged photoluminescence lifetime from T3 data}
In T3 mode, a correlation order \gn{1} is code for interactive-like behavior. Formally this is a correlation of the laser pulse and the system response, but this language is not often used.
\begin{verbatim}
> picoquant --file-in data.pt3 |  \
  histogram --mode t3 --order 1 --channels 2 \
            --time 0,10,500000
0,0.00,50000.00,0
0,50000.00,100000.00,0
...
1,0.00,50000.00,11267838
1,50000.00,100000.00,14947845
1,100000.00,150000.00,1512803
1,150000.00,200000.00,1152498
1,200000.00,250000.00,1037717
1,250000.00,300000.00,973572
1,300000.00,350000.00,932802
1,350000.00,400000.00,899615
1,400000.00,450000.00,12278
1,450000.00,500000.00,0
\end{verbatim}
Note that \texttt{histogram} will operate on channel 0 as well, even though channel 1 is the only channel with any signal. This costs extra memory and some computational overhead at startup, but ultimately the cost is insignificant compared to the cost of processing the data stream.

\subsubsection{\gn{2} from T2 data}
This data represents an electronic sync source (channel 4) and the detection of the laser itself.
\begin{verbatim}
> picoquant --file-in data.ht2 | \
  correlate --mode t2 --order 2 \
            --channels 5 \
            --max-time-distance 1000 | \
  histogram --mode t2 --order 2 \
            --channels 5 \
            --time 0,10,1000
0,0,0.00,100.00,0
...
0,4,0.00,100.00,43
0,4,100.00,200.00,24
0,4,200.00,300.00,38
0,4,300.00,400.00,43
0,4,400.00,500.00,44
...
4,4,900.00,1000.00,0
\end{verbatim}

\subsubsection{\gn{2} from T3 data}
\begin{verbatim}
> picoquant --file-in data.pt3 | \
  correlate --mode t3 --channels 2 \
            --order 2 \
            --max-pulse-distance 3 \
  histogram --mode t3 --channels 2 \
            --order 2 --pulse 0,3,3 \
            --time -500000,2,500000
0,0,0.00,1.00,-500000.00,0.00,0
...
1,1,0.00,1.00,-500000.00,0.00,0
1,1,0.00,1.00,0.00,500000.00,104640
1,1,1.00,2.00,-500000.00,0.00,123289
1,1,1.00,2.00,0.00,500000.00,124676
1,1,2.00,3.00,-500000.00,0.00,231962
1,1,2.00,3.00,0.00,500000.00,231694
\end{verbatim}
As a benchmark, on a computer with a dual-core 3\giga\hertz{} processor running 32-bit OpenSUSE 12.1, this command required 25\second{} of wall time, for \texttt{data.ht3} containing 32.7 million photon records (18.1\kilo\cps) for a total of 0.8 million correlation events.

\subsubsection{\gn{3} from T2 data}
This T2 data is the same as the laser data from before:
\begin{verbatim}
> picoquant --file-in data.ht2 | \
  correlate --mode t2 --order 2 \
            --channels 5 \
            --max-time-distance 1000000 | \
  histogram --mode t2 --order 2 \
            --channels 5 \
            --time 0,1,1000000
0,0,0.00,1000000.00,0,0.00,1000000.00,31
0,0,0.00,1000000.00,1,0.00,1000000.00,55
0,0,0.00,1000000.00,2,0.00,1000000.00,57
...
4,0,0.00,1000000.00,2,0.00,1000000.00,2710
4,0,0.00,1000000.00,3,0.00,1000000.00,2235
4,0,0.00,1000000.00,4,0.00,1000000.00,0
...
\end{verbatim}
Note how much larger the time window must be to catch significant numbers of higher-order events.

\subsection{Implementation details}
The problem of populating and returning histograms for all possible correlations can be broken into a few distinct steps:
\begin{enumerate}
\item Construct a histogram for every permutation of channels.
\item For a given correlation event:
  \begin{enumerate}
  \item Identify the histogram corresponding to the permutation of channels.
  \item Identify the bin associated with the parameters of the correlation.
  \item Increment that bin in that histogram.
  \end{enumerate}
\item For every bin of every histogram, print the bin definition and associated counts
\end{enumerate}
As such, each of these will be discussed separately.

\subsubsection{The histograms can be enumerated as a base-$N$ numbers}
Each histogram corresponds to a single cross-correlation identify by the vector $\vec{\channel}\in\channels^{n}$, for $\channels$ the set of all channels and correlation order $n$. As such, if we enumerate the channels as $\braces{0,1,\ldots N-1}$ and assign powers of $N$ to each index of the vector, we have found a way to uniquely identify the vector as an integer. If we index $\vec{\channel}=(\channel_{0},\channel_{1}\ldots)$, we can use the following function for enumeration:
\begin{equation}
\Index(\vec{c}) = \sum_{j=0}^{n-1}{c_{j}N^{n-j}}
\end{equation}
If $c_{j}\in\integers_{2}$ and $N=2$, this is identical to the conversion of $n$ binary digits to an integer. As is convention, we define the left-most digit ($\channel_{0}$) as carrying the highest power ($N^{n-1}$) and the right-most digit ($\channel_{n-1}$) as carrying the lowest power ($N^{0}=1$).

As an aside, note that this function is isomorphic (invertible), although we will never need to exploit this feature here. 

Thus, for any vector $\vec{c}$ encountered in the stream we have a method for identifying its corresponding cross-correlation and therefore the histogram it corresponds to. We also know there must be $N^{n}$ histograms, and therefore we pre-allocate them in memory before performing the actual processing of the stream. This leads to some amount of computational overhead for each instance of \histogram, but ultimately this will almost always be negligible in comparison with the cost of processing the stream. 

To see how this is implemented in \histogram, the functions \texttt{*combinations*} in \texttt{combinations.c} handle the base-$N$ number mathematics and the population of an array containing all $\vec{\channel}\in\channels$. These routines also determine the sorted list of indices and the order of indexing required to obtain the sort, as used for the \texttt{--channels-ordered} flag of \correlate. 

\subsubsection{A histogram is function of $n$-dimensional vectors mapping mapping onto integers}
Now that we can enumerate all histograms and identify the histogram some permutation of channels corresponds to, we can ignore the channels and focus on the timing (time or pulse) component of the correlation event. More specifically, each time dimension of the correlation is some value $t\in\integers$, and for each dimension we have specified some subset of $\integers$ to assign values to. Formally, we can assign any such $t$ to some value in $\integers_{a}$ for some $a\in\integers^{+}$. 

To see how this is performed, consider the role of the time limits $t_{\min}$ and $t_{\max}$. These define a range $[t_{\min},t_{\max})\subset\integers$ that $t$ is allowed to fall into (if $t\not\in[t_{\min},t_{\max})$ we reject it). Furthermore, we define a number of bins $\braces{b}=B$ which are themselves non-overlapping ranges in $[t_{\min},t_{\max})$ spanning the full range:
\begin{align}
\bigcup\limits_{b\in B}{b} &= [t_{\min},t_{\max}) \\
b_{j}\cap b_{k}&\not=\emptyset~\iff j=k
\end{align} 
As such, we can order and enumerate these bins by sorting their lower bound. This provides an isomorphic map from bin number to some index $j\in\integers_{\abs{B}}$, so any $t\in[t_{min},t_{max})$ can be mapped onto a value $\integers_{\abs{B}}$. 

Collecting all of these dimensions, we see that any given tuple of bin assignments $(a_{1},\ldots)$ is an element of bin space:
\begin{equation}
(a_{1},\ldots a_{n-1})\in B_{1}\times\ldots~B_{n-1}
\end{equation}
For this paper, the histogram counts the number of events associated with each such bin, so it is a function
\begin{equation}
\Histogram:B_{1}\times\ldots~B_{n-1}\rightarrow\wholes
\end{equation}
In principle the target set could also be \reals{} or something else, but \wholes{} is most appropriate here. Referring to equation~\ref{eq:gn_set}, we see that this function is exactly
\begin{equation}
\Histogram(a_{1},\ldots a_{n-1}) = \abs{\braces{(\photon_{0}.\ldots\photon_{n-1})
                                              \left|\begin{aligned}
                                              \gamma_{0}\in\photons_{c_{0}};\\ 
                                              \ldots;\\                                              
                                              \Time(\photon_{1})-\Time(\photon_{0})\in b_{a_{0}};\\
                                              \ldots
                                              \end{aligned}\right.}}
\end{equation}
where the ranges $b_{a_{j}}$ have replaced the implicit ranges $[\tau_{j},\tau_{j}+\epsilon_{c_{j}})$, and the channel definitions have been set previously. While we could normalize the result by the size of the bins $b_{a_{j}}$ here, this will be handled later to prevent precision issues.

From this, the algorithm for incrementing a bin is:
\begin{enumerate}
\item Draw a correlation event $g_{\photon}$ from the stream of correlation events. 
\item If $g_{\photon}$ does not exist, halt.
\item Determine the index of the histogram from $(\Channel(\photon_{0}), \Channel(\photon_{1}),\ldots\Channel(\photon_{n-1}))$.
\item Determine the bin indices $(a_{1},\ldots a_{n-1})$ from $(\tau_{1},\ldots\tau_{n-1})$ by binary searches over each dimension.
\item If any index $a_{j}$ is undefined, the event is not associated with \Histogram{} because some value is outside the valid range. Go to 1.
\item Increment the number of counts for the bin $\Histogram(a_{1},\ldots a_{n-1})$ and go to 1.
\end{enumerate}
In step 4, the binary search refers to the standard binary search algorithm. To see how this is performed, note that we have set $B$ of bins $b_{j}$ as defined above. We wish to associate some value $t$ with the bin $b_{a}$ such that $t\in b_{a}$. If no such $b_{a}\in B$ exists, we will indicate as such. Therefore, we are looking to search the tuple of indices $(0,1,\ldots \abs{B}-1)$ for the correct index as efficiently as possible, without any prior knowledge of the form of $b_{a}$. Such an algorithm is:
\begin{enumerate}
\item[0.] If $t\not\in[t_{\min},t_{\max})$, return $\emptyset$.
\item Set $a_{low}\leftarrow 0$, $a_{high}\leftarrow\abs{B}$.
\item Set $a\leftarrow\ceil{\frac{a_{high}-a_{low}}{2}}$
\item If $t\in b_{a}$, return $a$.
\item If $t>\max(b_{a})$, the index is too low. Set $a_{low}\leftarrow a$ and go to 2.
\item The index is too high. Set $a_{high}\leftarrow a$ and go to 2.
\end{enumerate}
Because the region of search halves with each successive step, this algorithm scales as O($\log(\abs{B})$). This could be reduced to O(1) if we require that the bins are linearly spaced, but for the moderate additional cost we maintain maximum flexibility of bin definitions. Given these considerations, we see that processing a correlation event costs O($n\log(\abs{B})$), which could be reduced to O($n$) if we restrict the bins to linear spacing\footnote{This factor of $n$ is really $2n$ for T3 mode, since we have pulse and time dimensions. This factor is left out here by convention but it is important to note that it does exist.}.

To see how this is implemented in \histogram, see \texttt{histogram\_gn.c} and \texttt{histogram\_t*.c}.

\subsubsection{Printing the histogram}
Having exhausted the stream of correlation events, the final task is to print all of the histogram bins in a readable format. This is done by iterating over the histograms, then iterating over the bins, printing the counts associated with each bin. How this process is performed should be evident from the preceding discussion, and for the details refer to the routine \texttt{print\_gn\_histogram} in \texttt{histogram\_gn.c}.

\section{GN}
\subsection{Purpose}
Given the $n$th-order histogram output from \histogram, \GN{} assembles the positive-time cross-correlations and produces all unique cross-correlations with negative time, in addition to the autocorrelation. These results are normalized to the bin width by default, but this behavior be suppressed with the flag \texttt{--no-bin-normalization}.

\subsection{Command-line syntax}
\subsection{Examples of usage}
\subsection{Implementation}

\section{Applications}
\subsection{Time-dependent photoluminescence lifetime}
\subsection{Bunching and antibunching}
\subsection{Fluorescence blinking}

\begin{appendix}
\section{Mathematical notation}
\label{sec:notation}
\subsection{Sets}

\subsection{Named sets}
\wholes, \integers
\end{appendix}

\end{document}
